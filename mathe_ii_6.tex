% !TeX spellcheck = de_DE
% !TEX encoding = UTF-8 Unicode
%%%%%%%%%%%%%%%%%%%%%%%%%%%%%%%%%%%%%%%%%%%%%%%%%%%%
% The first part of the header needs to be copied
%       into the note options in Anki.
%%%%%%%%%%%%%%%%%%%%%%%%%%%%%%%%%%%%%%%%%%%%%%%%%%%%

% layout in Anki:
\documentclass[9pt]{article}
\usepackage[a4paper]{geometry}
\geometry{paperwidth=.5\paperwidth,paperheight=100in,left=2em,right=2em,bottom=1em,top=2em}
\pagestyle{empty}
\setlength{\parindent}{0in}
 
% hyphenation:
\usepackage[ngerman]{babel}

% encoding:
\usepackage[T1]{fontenc}
\usepackage[utf8]{inputenc}
\usepackage{lmodern}

% packages:
\usepackage{parskip}
\usepackage[free-standing-units=true]{siunitx}
% Writng si units, numbers, list of numbers etc.
\usepackage{gensymb}
% Unified typseting of units outside of siuitx
\usepackage{amsmath}
\usepackage{amsfonts}  
% Math features
\usepackage{esdiff}
% Typeseting of (partial)derivatives 
\usepackage{commath}
% More  derivatives. Not so nice like esdiif but adds
% \dif comand for upright d in math mode.
\usepackage{bm}
% Bold font in math mode
\usepackage{esint}
% Some fancy integrals signs. Mutilpe integrals
\usepackage{enumerate}
% Different styles for enumerate lists
\usepackage{multirow}
% More advanced tabular
\usepackage{physics}
% Lots of usefull comannds for physicists. Vektors, nablas etc.
\usepackage{amssymb, xfrac, bbold}
% extra fonts and symbols 
\usepackage{mathtools}
% extension to amsmath, fixes, meany new tool
\usepackage{isotope}
\usepackage{empheq}
%commands:
\usepackage{ifthen}

\newcommand*{\tran}{^{\mkern-1.5mu\mathsf{T}}}
\newcommand{\Rn}{\mathbb{R}^n}
\newcommand{\Rk}{\mathbb{R}^k}
\newcommand{\R}[1]{%
	\ifthenelse{\equal{#1}{}}
	{\mathbb{R}}
	{\mathbb{R}^{#1}}}%
\newcommand{\C}[1]{%
	\ifthenelse{\equal{#1}{}}
	{\mathbb{C}}
	{\mathbb{C}^{#1}}}%	
\renewcommand{\vec}[1]{\underline{#1}}
\DeclarePairedDelimiter{\innerprod}\langle\rangle
\newcommand{\Fr}{\mathcal{F}}
\newcommand{\Hi}{\mathcal{H}}
\newcommand{\id}{\mathbb{1}}
\newcommand{\inv}{^{-1}}
%%%%%%%%%%%%%%%%%%%%%%%%%%%%%%%%%%%%%%%%%%%%%%%%%%%%
% Following part of header NOT to be copied into
%            the note options in Anki.
%          ! Anki will throw an errow !
%%%%%%%%%%%%%%%%%%%%%%%%%%%%%%%%%%%%%%%%%%%%%%%%%%%%%
%
%  pdf layout:
%
\geometry{paperheight=74.25mm}
\usepackage{pgfpages}
\pagestyle{empty}
\pgfpagesuselayout{8 on 1}[a4paper,border shrink=0cm]
\makeatletter
\@tempcnta=1\relax
\loop\ifnum\@tempcnta<9\relax
\pgf@pset{\the\@tempcnta}{bordercode}{\pgfusepath{stroke}}
\advance\@tempcnta by 1\relax
\repeat
\makeatother
% 
%  notes, fields, tags:
%
\newcommand{\xfield}[1]{
        #1\par
        \vfill
        {\tiny\texttt{\parbox[t]{\textwidth}{\localtag\\\globaltag\hfill\uuid}}}
        \newpage}
\newenvironment{field}{}{\newpage}
\newif\ifnote
\newenvironment{note}{\notetrue}{\notefalse}
\newcommand{\localtag}{}
\newcommand{\globaltag}{}
\newcommand{\uuid}{}
\newcommand{\tags}[1]{
    \ifnote 
        \renewcommand{\localtag}{#1}
    \else
        \renewcommand{\globaltag}{#1}
    \fi 
    }
\newcommand{\xplain}[1]{\renewcommand{\uuid}{#1}}
%
%%%%%%%%%%%%%%%%%%%%%%%%%%%%%%%%%%%%%%%%%%%%%%%%%%%%
% The following line again needs to be copied 
% into Anki:
\begin{document}
%%%%%%%%%%%%%%%%%%%%%%%%%%%%%%%%%%%%%%%%%%%%%%%%%%%%

\tags{mathe2::2sem::hilbertraum, beschrankte-lineare-operatoren}

\begin{note}
	\xplain{a095d0e5-ba05-4a86-8601-7c56267afc34}
	\tags{}
	
	\begin{field}  % Question
		Wann ist eine linearer Operator aus $\Hi$ \textit{dicht definiert}?
	\end{field}
	
	\begin{field}  % Answer
		Wenn $D(A)$ \textit{Definitionsbereich} dicht in $\Hi$ ist.
	\end{field}
\end{note}
	\begin{note}
		\xplain{8220fc0f-51a5-419d-8b5b-a7a6c8241a58}
		\tags{6.1.3}
		
		\begin{field}  % Question
			Wie ist ein beschr"ankter Operator definiert?	
		\end{field}
		
		\begin{field}  % Answer
			Sei $A: D(A) \rightarrow \Hi$ ein linearer Operator. $A$ hei"st beschr"ankt, falls
			f"ur jede beschr"ankte Menge $M\subset D(A)$ gilt, dass $A[M]$ beschr"ankt ist.
		\end{field}
		
			
		\begin{field}  % Question
			Sei $A: D(A) \rightarrow \Hi$ ein linearer Operator. $A$ hei"st stetig, falls
		\end{field}
		
		\begin{field}  % Answer
			f"ur jede Folge $(\varphi_n)$ in $D(A)$ mit $\varphi_n \rightarrow \varphi \in D(A)$ gilt, dass
			\begin{equation*}
				A\varphi_n \rightarrow A\varphi
			\end{equation*}
		\end{field}
	\end{note}
	\begin{note}
		\xplain{c8e07072-d74a-4c22-97cf-551bf429f5d6}
		\tags{6.1.4, satz}
		
		\begin{field}  % Question
			Sei $A:D(A) \rightarrow \Hi$ ein linearer Operator. Welche Aussagen sind "aquivalent zu
			\begin{itemize}
				\item $A$ ist beschr"ankt ?
			\end{itemize}
		\end{field}
		
		\begin{field}  % Answer
			\begin{itemize}
				\item $\exists c>0 : \forall \varphi \in D(A) \norm{A\phi} \leq c\norm{phi}$,
				\item $A$ ist stetig,
				\item $A$ ist stetig in 0.
			\end{itemize}
		\end{field}
		
		\begin{field}  % Question
			Wann l"asst sich ein linearer Operator $A$ auf ganz $\Hi$ fortsetzen?
		\end{field}
		
		\begin{field}  % Answer
			Falls $D(A)$ dicht in $\Hi$ und $A$ stetig, l"asst sich $A$ \textbf{stetig} auf $\Hi$ fortsetzen.
		\end{field}
	\end{note}
	\begin{note}
		\xplain{62c8419e-280e-458d-bd7f-fe2a0f0f0ddd}
		\tags{6.1.5, 6.1.6, 6.1.7, operatornorm}
		
		\begin{field}  % Question
			Was ist $B(\Hi)$? 
		\end{field}
		
		\begin{field}  % Answer
			$B(\Hi)$ ist der Vektorraum aller \textbf{beschr"ankten} linearen Operatoren auf $\Hi$.
			Es ist ein Vektorraum ,da f"ur ein skalar $c$ und zwei beschr"ankte Operatoren $A_1, A_2$
			$A_1 + A_2$ und $cA$ auch beschr"ankt sind. 
		\end{field}
		
			
		\begin{field}  % Question
			Wie ist die Operatornorm auf $B(\Hi)$ definiert? 
		\end{field}
		\begin{field}  % Answer
			\begin{equation*}
				\norm{T} \coloneqq \ \text{inf}\qty{c>0 \ | \ \forall \varphi \in \Hi: \norm{T\varphi} \leq c\norm{\varphi}}
			\end{equation*}
		\end{field}
			
		\begin{field}  % Question
			Was besagt der Satz Norm auf $B(\Hi)$, au"ser dass die definierte Operatornorm tats"achlich einer Norm auf $(\Hi)$ ist? 
		\end{field}
		
		\begin{field}  % Answer
			$\qty(B(\Hi), \norm{\cdot})$ ist ein Banach-Raum.
			
			Es gilt die \textit{Submultiplikativit"at} der Norm, also $\norm{AB} \leq \norm{A}\norm{B}$.
		\end{field}
			
		\begin{field}  % Question
			Welche "aquivalente zu 
			\begin{equation*}
			\norm{T} \coloneqq \ \text{inf}\qty{c <0 \ | \ \forall \varphi \in \Hi
			\norm{T\varphi} \leq \norm{\varphi}}
			\end{equation*}
		 Charakterisierungen der Operatornorm auf $B(\Hi)$ gibt es (Es sind 4.)?
		\end{field}
		
		\begin{field}  % Answer
			\begin{align*}
				\norm{T} =& \ \text{sup}\qty{\norm{T\varphi} \ | \ \varphi \in \Hi, \norm{\varphi} \leq 1} \\
				\norm{T} =& \ \text{sup}\qty{\norm{T\varphi} \ | \ \varphi \in \Hi, \norm{\varphi} = 1} \\
				\norm{T} =& \ \text{sup}\qty{\norm{T\varphi} \ | \ \varphi \in \Hi, \norm{\varphi} < 1} \\
				\norm{T} =& \ \text{sup}\qty{\abs{\innerprod{\psi, T\varphi}} \ | \ \psi, \varphi \in \Hi, \qty(\norm{\psi} \leq 1 \wedge \norm{\varphi} \leq 1)} \\
			\end{align*}
		\end{field}
	\end{note}
	\begin{note}
		\xplain{b514ddca-b0be-4779-b1ea-10bc1ff4d0f9}
		\tags{6.1.8}
		
		\begin{field}  % Question
			Ein Operator aus $B(\Hi)$ hei"st \textit{endlich-dimensional}, wenn ...
		\end{field}
		
		\begin{field}  % Answer
			... der Bildbereich von diesem Operator endlich-dimensional ist.
		\end{field}
	\end{note}
	\begin{note}
		\xplain{0024bfb9-70b6-40a3-8cb4-8675d0a285cd}
		\tags{6.1.9,6.1.10, inverse}
		
		\begin{field}  % Question
			Wann existiert der inverse Operator zu einem linearen Operator $A$?  
		\end{field}
		
		\begin{field}  % Answer
			Falls $A: D(A) \rightarrow \text{im}A$ injektiv. Dann ist $A^{-1}$ auch linear.
		\end{field}
			
		\begin{field}  % Question
			Was besagt der Satz "uber den inversen Operator?
		\end{field}
		
		\begin{field}  % Answer
			Wenn $A\in B(\Hi), \ \text{im}A=\Hi$ und $A^{-1}$ existiert, dann ist $A^{-1} \in B(\Hi)$.
		\end{field}
	\end{note}
	\begin{note}
		\xplain{d5cd0fc1-fa83-416d-b72f-f89f21b26f67}
		\tags{6.2.1, adjungierte-operator}
		
		\begin{field}  % Question
			Was besagt der Sagt "uber den adjungierten Operator?
		\end{field}
		
		\begin{field}  % Answer
			Zu jedem $T \in B(\Hi)$ existiert genau ein $T^*\in B(\Hi)$ mit folgender Eigenschaft:
			\begin{equation*}
				\forall \varphi, \psi \in \Hi: \quad \innerprod{\varphi, T\psi} = \innerprod{T^*\varphi, \psi}
			\end{equation*}
			$T^*$ hei"st der zu $T$ adjungierte Operator. Es gilt: $\norm{T}= \norm{T^*}$
		\end{field}
	\end{note}
	\begin{note}
		\xplain{3e608f64-383c-47f7-a110-70cb0066c5e9}
		\tags{6.2.2}
		
		\begin{field}  % Question
			$(B(\Hi), \norm{\cdot})$ ist eine $C^*$-Algebra mit Eins-Element, d.h. ...
		\end{field}
		
		\begin{field}  % Answer
			...
			\footnotesize
			\begin{enumerate}[i)]
				\item $B(\Hi)$ ist ein $\C{}$-Vektorraum mit einer \textbf{Multiplikation} (assoziative, bilineare Abbildung)
				\begin{equation*}
					B(\Hi) \times B(\Hi) \ni (S,T) \mapsto S \circ T
				\end{equation*}
				Als Vektorraum mit \textbf{assoziativer bilinearer} Abbildung 
				ist $B(\Hi)$ eine Abbildung. Das Eins-Element dieser Algebra ist Einheitsoperator $\mathbb{1}$.
				\item $(B(\Hi), \norm{\cdot})$ ist eine \textbf{normierte} Algebra, d.h. $\norm{\cdot}$ ist \textbf{submultiplikativ}:
				\begin{equation*}
					\forall S,T \in B(\Hi): \norm{ST} \leq \norm{S} \norm{T}
				\end{equation*}
				Es ist eine \textbf{Banach-Algebra} (also vollst"andig und normiert).
				\item $B(\Hi)$ ist eine \textbf{$^*$-Algebra}.
				\item Die Norm erf"ullt die \textbf{${C^*}$-Algebra}, d.h. $\forall T \in B(\Hi): \norm{T^*T} = \norm{T}^2$.
			\end{enumerate}
		\end{field}
			
		\begin{field}  % Question
			$B(\Hi)$ ist eine \textbf{$^*$-Algebra}, d.h. ...
		\end{field}
		
		\begin{field}  % Answer
			... es gibt eine Abbildung $*: B(\Hi) \rightarrow B(\Hi), \ \ T\mapsto T^*$ mit
			
			$\forall S,T \in B(\Hi), \forall \lambda \in \C{}:$
			\begin{enumerate}
				\item $\qty(\lambda S + T)^* = \bar{\lambda}S^* + T^*$,
				\item $\qty(ST)^* = T^*S^*$,
				\item $S^{**} = S$
			\end{enumerate}
		Eine solche Abbildung hei"st \textit{Involution}.
		\end{field}
	\end{note}
	\begin{note}
		\xplain{73250023-243d-431b-9f96-3cebc5b78556}
		\tags{6.2.5}
		
		\begin{field}  % Question
			Wie lautet der Satz "uber das orthogonale Komplement des Bildes?
		\end{field}
		
		\begin{field}  % Answer
			Sei $T\in B(\Hi)$. Dann gilt
			\begin{align*}
				\qty(\text{im} T)^\perp =& \ \text{ker}T^*\\
				\qty(\text{im} T^*)^\perp =& \ \text{ker}T
			\end{align*}
		\end{field}
	\end{note}
	\begin{note}
		\xplain{f5c2e986-17a9-48b7-9d81-a42de0547888}
		\tags{}
		
		\begin{field}  % Question
			Ein Operator $T\in B(\Hi)$ hei"st \textit{selbstadjungiert}, wenn...
		\end{field}
		
		\begin{field}  % Answer
		... $T=T^*$
		\end{field}
	
				\begin{field}  % Question
			Ein Operator $T\in B(\Hi)$ hei"st \textit{positiv}, wenn...
		\end{field}
		
		\begin{field}  % Answer
			...$\forall \varphi \in \Hi: \innerprod{T\varphi, \varphi} \geq 0$
		\end{field}
	
				\begin{field}  % Question
			Ein Operator $T\in B(\Hi)$ hei"st \textit{unit"ar}, wenn...
		\end{field}
		
		\begin{field}  % Answer
			... $T^{-1} = T^*$
		\end{field}
	
				\begin{field}  % Question
			Ein Operator $T\in B(\Hi)$ hei"st \textit{normal}, wenn...
		\end{field}
		
		\begin{field}  % Answer
			... $T^*T=TT^*$
		\end{field}
	
				\begin{field}  % Question
			Ein Operator $T\in B(\Hi)$ hei"st \textit{Projektion}, wenn...
		\end{field}
		
		\begin{field}  % Answer
			... $T^2=T$
		\end{field}
	
				\begin{field}  % Question
			Ein Operator $T\in B(\Hi)$ hei"st \textit{}, wenn...
		\end{field}
		
		\begin{field}  % Answer
			...
		\end{field}
	
				\begin{field}  % Question
			Ein Operator $T\in B(\Hi)$ hei"st \textit{Orthogonalprojektion}, wenn...
		\end{field}
		
		\begin{field}  % Answer
			... $T$ Projektion ist und $T=T^*$
		\end{field}
		\begin{field}  % Question
			Ein Operator $T\in B(\Hi)$ hei"st \textit{isometrich}, wenn...
		\end{field}
	
		\begin{field}  % Answer
			... $\forall \varphi \in \Hi: \norm{T\varphi} = \norm{T}$
		\end{field}
		\begin{field}  % Question
			Ein Operator $T\in B(\Hi)$ hei"st \textit{partiell isometrisch}, wenn...
		\end{field}

		\begin{field}  % Answer
			... eine Zerlegung $\Hi = \Hi_1 \oplus \Hi_2$ existiert mit $T:\Hi_1\rightarrow\Hi$ ist isometrisch
			und $\Hi_2 = \ \text{ker}T$.
			
			$\Hi_1$ ist der \textit{Anfangsbereich}  und $\text{im}T$ der \textit{Endbereich} der partiellen Isometrie. 
		\end{field}
	\end{note}
	\begin{note}
		\xplain{a16a3dd4-e1ab-4e92-a6bd-a22a39844a31}
		\tags{6.2.7}
		
		\begin{field}  % Question
			Wann existiert ein - zum unbeschr"ankten Operator - adjungierte Operator?
		\end{field}
		
		\begin{field}  % Answer
			Wenn der Operator dicht definiert ist.
		\end{field}
			
		\begin{field}  % Question
			Sei $T:D(T)\rightarrow \Hi$ ein dicht definierter Operator. Es existiert also der zu $T$ adjungierte Operator 
			$T^*:D(T^*) \rightarrow \Hi$. Wie ist $T^*$ definiert? 
		\end{field}
		
		\begin{field}  % Answer
			$D(T*) = D^*\coloneqq \qty{\psi \in \Hi \ | \ D(T) \ni \mapsto \innerprod{\psi, T\varphi} \in \C{} \text{ ist stetig}}$
			
			$\forall \varphi \in D(T) \forall \psi \in D^*: \innerprod{T^*\psi, \phi} = \innerprod{\psi, T\phi}$			
		\end{field}
		
	\begin{field}  % Question
		Welche wichtige Bemerkungen gibt es zu   unbeschr"ankten adjungierten Operatoren?
	\end{field}
	
	\begin{field}  % Answer
		\begin{itemize}
			\item Aus "$T$ dicht definiert"  folgt \textbf{nicht}, dass auch $T^*$ dicht definiert!
			\item Falls $T^*$ nicht dicht definiert ist, kann man $T^{**}$ 	nicht eindeutig definieren.
			\item Auch wenn $D(T^*)$ dicht in $\Hi$ gilt nicht unbedingt, dass $T^{**} = T$!!
		\end{itemize}
	\end{field}
	\end{note}
	\begin{note}
		\xplain{21f44b91-9d51-45bb-b869-36f9fbb8415b}
		\tags{6.2.9}
		
		\begin{field}  % Question
			Was bedeuten (und was sind die Unterschiede): \textit{hermitesch}, \textit{symmetrich} und \textit{selbstadjungiert}, f"ur unbeschr"ankte Operatoren?
			
		\end{field}
		
		\begin{field}  % Answer
			\begin{itemize}
				\item \textit{hermitesch} oder \textit{formal adjungiert}: $\forall \varphi, \psi \in D(T): \innerprod{T\psi,\varphi}
				= \innerprod{\psi, T\varphi}$. $T$ muss hier \textbf{nicht} dicht definiert sein!
				\item \textit{symmetrisch}: $T$ hermitesch und dicht definiert. Also es gilt insbesondere $T\subset T^*$.
				\item \textit{selbstadjungiert} :$T$ dicht definiert und sowohl $D(T)\subset D(T^*)$, als auch $D(T^*) \subset D(T)$,
				also $T^*=T$ gilt. Insbesondere $D(T^*) = D(T)$ und $T$ \textit{symmetrisch}.
				
			\end{itemize}
		\end{field}
	\end{note}
	\begin{note}
		\xplain{e1ec3a29-1efb-4c61-8604-06bf284919cf}
		\tags{6.3, spektrum}
		
		\begin{field}  % Question
			Wie ist die \textit{Resolventenmenge} un das \textit{Spektrum} eines beschr"ankten Operators definiert?
		\end{field}
		
		\begin{field}  % Answer
			Sei $T \in B(\Hi)$,
			\begin{enumerate}[i)]
				\item Die Menge 
				\begin{equation*}
					\varrho(T) \coloneqq \qty{\lambda \in C{} \ | \ \exists \qty(T-\lambda \id)\inv \in B(\Hi)}
				\end{equation*}
				hei"st \textit{Resolventenmenge von $T$}, f"ur $\lambda \in \varrho(T)$ gilt also: 
				$(T-\lambda \id): \Hi \rightarrow \Hi$ ist bijektiv.
				
				Der Operator $R_\lambda(T) \coloneqq(T-\lambda\id)\inv$ hei"st \textit{Resolvente von $T$ im Punkt $\lambda$}.
				\item Die Menge $\sigma(T)\coloneqq \C{}\setminus\varrho(T)$ hei"st \textit{Spektrum von $T$}.
			\end{enumerate}
		\end{field}
	\end{note}
	\begin{note}
		\xplain{03cf9f0b-a62c-447a-97f2-bd6fe8a534e7}
		\tags{6.3.3 Resolventenmenge}
		
		\begin{field}  % Question
			Was sind die Aussagen des Satzes "uber die Eigenschaften der Resolventenmenge?
		\end{field}
		
		\begin{field}  % Answer
			\footnotesize
			\begin{enumerate}[i)]
				\item F"ur $ \lambda, \mu \in \varrho(T)$ kommutieren die Operatoren $R_\lambda(T)$ und $R_\mu(T)$ und es gilt die \textit{Resolventengleichung}
				\begin{equation*}
					R_\lambda(T) - R_\mu(T) = (\lambda - \mu)R_\lambda(T) \cdot R_\mu(T)
				\end{equation*}
				\item F"ur $\lambda$ mit $\abs{\lambda} > \norm{T}$ gilt $\lambda \in \varrho(T)$. $R_\lambda(T)$ wird durch die
				\textbf{\textit{Neumannsche Reihe}} beschrieben
				\begin{equation*}
					R_\lambda(T) = - \sum^\infty_{k=0} \frac{T^k}{\lambda^{k+1}}
				\end{equation*}
				Es gilt die Absch"atzung $\norm{\qty(T-\lambda\id)\inv} \leq \frac{1}{\abs{\lambda} - \norm{T}}$.
				\item F"ur $\lambda_0 \in \varrho(T)$ gilt: $\lambda \in \C{}, \abs{\lambda - \lambda_0}<\norm{R_{\lambda_0}}\inv \Rightarrow$ die Reihe
				\begin{equation*}
					R_{\lambda_0}(T) \qty[\id + \sum^\infty_{k=0}(\lambda - \lambda_0)^k \qty(R_{\lambda_0}(T))^k]
				\end{equation*}
				konvergiert und es ist gleich $R_\lambda(T)$. 	Also $\lambda \in \varrho(T)$ und $\varrho(T)$ ist offen.
			\end{enumerate}
		\end{field}
	\end{note}
	\begin{note}
		\xplain{bb06deb4-96be-4cad-ac9f-d46551679abd}
		\tags{6.3.4}
		
		\begin{field}  % Question
			Was besagt der Satz "uber die Kompaktheit des Spektrums?
		\end{field}
		
		\begin{field}  % Answer
			Sei $T\in B(\Hi)$. Das Spektrum $\sigma(T)$ ist eine kompakte, nichtleere Teilmenge von $\C{}$. Es gilt
			$\sigma(T) \subset 
			qty{\lambda \in \C{} \ | \ \abs{\lambda} \leq \norm{T}}$
		\end{field}
	\end{note}
	\begin{note}
		\xplain{471d3474-c216-46e9-b61c-097278aa2327}
		\tags{6.3.5}
		
		\begin{field}  % Question
			Wie ist ein Punktspektrum $\sigma_p$ definiert?
		\end{field}
		
		\begin{field}  % Answer
			\begin{align*}
				\sigma_p \coloneqq& \qty{\lambda \in \C{} \ | \ (T-\lambda\id) \text{ nicht injektiv}} \\
						 \coloneqq& \qty{\lambda \in \C{} \ | \ \exists \varphi \in \Hi\setminus \qty{0}: T\varphi = \lambda \varphi} 
			\end{align*}
		\end{field}
			
		\begin{field}  % Question
			Wie ist das stetige Spektrum eines Operators $T$ definiert?
		\end{field}
		
		\begin{field}  % Answer
			\begin{align*}
				\sigma_c(T) \coloneqq& \big\{\lambda \in \C{} \ | \ (T-\lambda \id) \text{ injektiv, nicht surjektiv, } \\
										    &\quad(T-\lambda\id)[\Hi] \text{ dicht in } \Hi\big\} \\
							\coloneqq& \big\{ \lambda \in \C{} \ | D\coloneqq (T-\lambda\id)[\Hi] \text{ dicht in } \Hi, \\
											&\quad\exists(T-\lambda\id)\inv : D \rightarrow \Hi,  \text{ nicht dicht in } \Hi \big\}
			\end{align*}
		\end{field}
			
		\begin{field}  % Question
			Wie ist das Restspektrum von einem Operator $T$ definiert?
		\end{field}
		
		\begin{field}  % Answer
			\begin{align*}
			\sigma_c(T) \coloneqq& \big\{\lambda \in \C{} \ | \ (T-\lambda \id) \text{ injektiv, } \\
			&\quad(T-\lambda\id)[\Hi] \text{ nicht dicht in } \Hi\big\} \\
			\coloneqq& \qty{\lambda \in \C{} \ | \ \lambda \notin \sigma_p, (T-\lambda\id)[\Hi] \text{ nicht dicht in }
				\Hi}
			\end{align*}
		\end{field}
	\end{note}
	\begin{note}
		\xplain{c8b5665b-208a-481b-bd1b-b865b4bbdfc8}
		\tags{6.3.6}
		
		\begin{field}  % Question
			Zusammenhang Spektrum adjungierter und nicht adjungierter beschr"ankten Operatoren:
		\end{field}
		
		\begin{field}  % Answer
			\begin{align*}
				\sigma(T^*) =& \overline{\sigma(T)} \\
				\varrho(T^*) =& \overline{\varrho(T)}
			\end{align*}
			hier $\overline{\quad}$ c.c. der Elemente.
			\begin{equation*}
				R_{\bar{\lambda}}(T^*)= R_\lambda(T)
			\end{equation*}
			\begin{align*}
				\lambda \in \sigma_p &\Rightarrow \bar{\lambda} \in \sigma_p(T*) \cup \sigma_r(T^*) \\
				\lambda \in \sigma_r(T) & \Rightarrow \bar{\lambda} \in \sigma_p(T^*) \\
				\lambda \in \sigma_c(T) & \Rightarrow \bar{\lambda} \in \sigma_c(T^*) 
			\end{align*}
		\end{field}
	\end{note}
	\begin{note}
		\xplain{fbceb1aa-ca13-4f80-ac7e-dfcdc6b9b554}
		\tags{6.4}
		
		\begin{field}  % Question
			Was gilt f"ur Spektrum im Spezialfall der \textbf{selbstadjungierten} beschr"ankten Operatoren? 
		\end{field}
		
		\begin{field}  % Answer
			Sei $T$ =$T^* \in B(\Hi)$. Dann gilt:
				\begin{enumerate}[i)]
					\item $\sigma_r(T) = \emptyset$,
					\item f"ur $\lambda \in \C{}$ gilt:
					\begin{equation*}
						\lambda \in \varrho(T) \Leftrightarrow \exists c>0: \forall \varphi \in \Hi: \norm{\qty(T-\lambda\id)\varphi}
						\geq c\norm{\varphi}
					\end{equation*}
					\item Weylsches Kriterium: F"ur $\lambda \in \C{}$ gilt:
						\begin{align*}
							\lambda \in \sigma(T) \Leftrightarrow& \exists \text{ Folge } (\varphi_n) \in \Hi \qty(\text{ mit }
							\forall n \in \N{}: \norm{\varphi_n}=1) \\ &\qquad\text{ und }\norm{(T-\lambda \id)\varphi_n} \overset{n\rightarrow \infty}{\rightarrow} 0 
						\end{align*}
						\item $\sigma(T)\subset \R{}$ und die Eigenvektoren zu paarweise verschiedenen Eigenwerten stehen orthogonal aufeinander. 
				\end{enumerate}
		\end{field}
	\end{note}
	\begin{note}
		\xplain{64f8c427-8cfb-487e-a0f3-d9878517ecd7}
		\tags{6.4.2}
		
		\begin{field}  % Question
			Wie lautet der Satz "uber den Rand des Spektrums?
		\end{field}
		
		\begin{field}  % Answer
			Sei $T=T^* \in B(\Hi)$. Dann gilt $\norm{T} \in \sigma(T)$ oder $-\norm{T} \in \sigma(T)$.
		\end{field}
	\end{note}
\begin{note}
	\xplain{d3eb783b-ec94-4980-8b2a-f7a76018c965}
	\tags{6.4.3}
	
	\begin{field}  % Question
		Wie lautet das Lemma "uber die Darstellung von $\abs{T}$, falls $T$ ein selbstadjungierter Operator?
	\end{field}
	
	\begin{field}  % Answer
		Seit $T=T^* \in B(\Hi)$. Dann gilt:
		\begin{equation*}
			\norm{T}= \text{sup}\qty{\abs{\innerprod{\varphi,T\varphi}} \ | \ \norm{\varphi} \leq 1}
		\end{equation*}
	\end{field}
\end{note}
\begin{note}
	\xplain{c394653d-6873-4c47-9291-141276a64b82}
	\tags{7.1}
	
	\begin{field}  % Question
		Sei $\Hi$ ein Hilbertraum. Eine Menge $M\subset\Hi$ hei"st (folgen-) kompakt, wenn 
	\end{field}
	
	\begin{field}  % Answer
		jede Folge $(\psi_n)_{n\in \N{}}$ in $m$ ein Teilfolge $(\psi_nk)_{k\in \N{}}$ enth"alt, die gegen ein Element aus $M$ konvergiert. 
	\end{field}
		
	\begin{field}  % Question
		Sei $\Hi$ ein Hilbertraum. Eine Menge $M\subset\Hi$ hei"st relativ kompakt, wenn 
	\end{field}
	
	\begin{field}  % Answer
		der Abschluss $\overline{M}$ kompakt ist, mit anderen Worten, jde Folge in $M$ enth"alt eine Teilfolge die gegen ein Element aus $\Hi$ konvergiert. 
	\end{field}
\end{note}
\begin{note}
	\xplain{22856de0-3411-435a-9a10-a4dbb48ee976}
	\tags{7.1.2}
	
	\begin{field}  % Question
    	Ein linearer Operator $T: \Hi \rightarrow \Hi$ hei"st kompakt, 
	\end{field}
	
	\begin{field}  % Answer
		wenn f"ur jede beschr"ankte Menge $M\subset\Hi$ gilt, dass
		$T[M]$ eine relativ kompakte Menge ist.
	\end{field}
\end{note}
\begin{note}
\xplain{a0f8c202-a1ca-4741-ac92-5460c7f963c2}
\tags{}

\begin{field}  % Question
	$K(\Hi)$ ist ein... in $B(\Hi)$.
\end{field}

\begin{field}  % Answer
	abgeschlossenes zwei-seitiges $^*$-Ideal
\end{field}

\begin{field}  % Question
	$K(\Hi)$ ist eine abgeschlossenes zwe-seitiges $^*$-Ideal, d.h. ...
\end{field}

\begin{field}  % Answer
	...
	\begin{enumerate}[a)]
		\item \label{1} $K(\Hi)$ ist ein Vektorraum.
		\item \label{2} $\forall (T_n) \in K(\Hi)$ mit $\exists T \in B(\Hi), T_n \rightarrow T$ (bzgl. der Operatornorm) gilt: $T\in K(\Hi)$.
		\item \label{3} Idealeigenschaft: $\qty(T\in K(\Hi) \wedge S \in B(\Hi)) \Rightarrow$ \\ 
		\hspace*{90pt}$\Rightarrow\qty(ST \in K(\Hi) \wedge TS \in K(\Hi))$.
		\item \label{4} $T\in K(\Hi) \Rightarrow T^* \in K(\Hi)$
		
		Ideal = \ref{1} + \ref{3}; *-Ideal = Ideal + \ref{4}; Abgeschlossenheit = \ref{2}, Zwei-Seitigkeit = \ref{3} f"ur TS \textbf{und} ST.
	\end{enumerate}	
\end{field}
	
\begin{field}  % Question
	Welche bezielhung gilt zwischen $F(\Hi)$, also den Raum der endlich-dimensionalen Operatoren, und $K(\Hi)$?
\end{field}

\begin{field}  % Answer
	\begin{itemize}
		\item $F(\Hi) \subseteq K(\Hi)$
		\item $F(\Hi)$ liegt dicht in $K(\Hi)$
	\end{itemize}
\end{field}
\end{note}
\begin{note}
	\xplain{f523d7a1-96ed-4dad-addb-b26c2802745f}
	\tags{7.2.1}
	
	\begin{field}  % Question
		Was besagt der Satz "uber das Spektrum der kompakten selbstadjungierten Operatoren?
	\end{field}
	
	\begin{field}  % Answer
		Sie $\Hi$ ein unendlich-dimensionaler Hilbertraum und $T=T^* \in K(\Hi)$. Dann gilt: 
		\begin{enumerate}[i)]
			\item $0 \in \sigma(T)$,
			\item jedes $\lambda \in \sigma(T)$ ist ein EIgenwert endlicher Vielfachheit,
			\item ist $T$ nicht von endlichem Rang, so bilden die Eigenwerte von $T$ eine Nullfolge.
		\end{enumerate}
	\end{field}
\end{note}
\begin{note}
	\xplain{fb2138d7-0fbb-455c-9d0d-7266a21c89d4}
	\tags{7.2.2}
	
	\begin{field}  % Question
		Was besagt der Spektraltheorem f"ur kompakte selbstadjungierte Operatoren?
	\end{field}
	
	\begin{field}  % Answer
		Sei $T=T^*> \in K(\Hi), \lambda_1 \lambda_2 \cdots$ seien die von Null verschiedenen Eigenwerte. Sei $\abs{\lambda_1} \geq \abs{\lambda_2}\geq \dots$. Seien $P_i$ die (endlich-dimensionalen) orthogonalen Projektionen auf die Eigenr"aume zu $\lambda_i$. Dann gilt:
		\begin{equation*}
			T = \sum_{j} \lambda_jP_j
		\end{equation*}
		Falls T nicht endlich-dimensional ist (d.h. abz"ahlbar viele Eigenwerte ungleich Null), dann konvergiert die unendliche Reihe bzgl. der Operatornorm.
	\end{field}
\end{note}
\begin{note}
	\xplain{27aa296d-0be5-4094-a837-a0c32a57d34e}
	\tags{7.2.3}
	
	\begin{field}  % Question
		Wie lautet der Hilbert-Schmidtscher Entwicklungssatz?
	\end{field}
	
	\begin{field}  % Answer
		Sei $T=T^* \in K(\Hi)$. Dann existiert eine Folge $(\mu_i)$ in $\R{}$ (endlich oder Nullfolge) und ein Orthonormalsystem $(\phi_i)$ in $\Hi$ so, dass
		\begin{equation*}
			\forall \varphi in \Hi: \quad T\varphi = \sum_i \mu_i \innerprod{\varphi_i, \varphi}\varphi_i
		\end{equation*}
	\end{field}
\end{note}
\begin{note}
	\xplain{4f7ece72-2a59-4131-b755-2e54e0310ecb}
	\tags{7.3.1}
	
	\begin{field}  % Question
		Wie ist ein Spurklassenoperator definiert?
	\end{field}
	
	\begin{field}  % Answer
		Ein kompakter, positiver, selbstadjungierter Operator $T: \Hi \rightarrow \Hi$ hei"st \textit{nuklear} oder \textit{Spurklassenoperator}, wenn f"ur eine ONB $(\psi_k)$ von $\Hi$ gilt:
		\begin{equation*}
			\sum _{k=1}^\infty \innerprod{\psi_k, T\psi_k} < \infty
		\end{equation*}
	\end{field}
		
	\begin{field}  % Question
		Ein Spurklassenoperator hei"st \textit{ Dichteoperator}, wenn:
	\end{field}
	
	\begin{field}  % Answer
		\begin{equation*}
			\sum_{k=1}^\infty \innerprod{\psi_k, T\psi_k} = 1
		\end{equation*}
		$(\psi_k)$ eine ONB.
	\end{field}
\end{note}
\end{document}
