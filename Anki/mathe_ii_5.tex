% !TeX spellcheck = de_DE
% !TEX encoding = UTF-8 Unicode
%%%%%%%%%%%%%%%%%%%%%%%%%%%%%%%%%%%%%%%%%%%%%%%%%%%%
% The first part of the header needs to be copied
%       into the note options in Anki.
%%%%%%%%%%%%%%%%%%%%%%%%%%%%%%%%%%%%%%%%%%%%%%%%%%%%

% layout in Anki:
\documentclass[9pt]{article}
\usepackage[a4paper]{geometry}
\geometry{paperwidth=.5\paperwidth,paperheight=100in,left=2em,right=2em,bottom=1em,top=2em}
\pagestyle{empty}
\setlength{\parindent}{0in}
 
% hyphenation:
\usepackage[ngerman]{babel}

% encoding:
\usepackage[T1]{fontenc}
\usepackage[utf8]{inputenc}
\usepackage{lmodern}

% packages:
\usepackage{parskip}
\usepackage[free-standing-units=true]{siunitx}
% Writng si units, numbers, list of numbers etc.
\usepackage{gensymb}
% Unified typseting of units outside of siuitx
\usepackage{amsmath}
\usepackage{amsfonts}  
% Math features
\usepackage{esdiff}
% Typeseting of (partial)derivatives 
\usepackage{commath}
% More  derivatives. Not so nice like esdiif but adds
% \dif comand for upright d in math mode.
\usepackage{bm}
% Bold font in math mode
\usepackage{esint}
% Some fancy integrals signs. Mutilpe integrals
\usepackage{enumerate}
% Different styles for enumerate lists
\usepackage{multirow}
% More advanced tabular
\usepackage{physics}
% Lots of usefull comannds for physicists. Vektors, nablas etc.
\usepackage{amssymb, xfrac}
% extra fonts and symbols 
\usepackage{mathtools}
% extension to amsmath, fixes, meany new tool
\usepackage{isotope}
\usepackage{empheq}
%commands:
\usepackage{ifthen}

\newcommand*{\tran}{^{\mkern-1.5mu\mathsf{T}}}
\newcommand{\Rn}{\mathbb{R}^n}
\newcommand{\Rk}{\mathbb{R}^k}
\newcommand{\R}[1]{%
	\ifthenelse{\equal{#1}{}}
	{\mathbb{R}}
	{\mathbb{R}^{#1}}}%
\newcommand{\C}[1]{%
	\ifthenelse{\equal{#1}{}}
	{\mathbb{C}}
	{\mathbb{C}^{#1}}}%	
\renewcommand{\vec}[1]{\underline{#1}}
\DeclarePairedDelimiter{\innerprod}\langle\rangle
\newcommand{\Fr}{\mathcal{F}}
\newcommand{\Hi}{\mathcal{H}}

%%%%%%%%%%%%%%%%%%%%%%%%%%%%%%%%%%%%%%%%%%%%%%%%%%%%
% Following part of header NOT to be copied into
%            the note options in Anki.
%          ! Anki will throw an errow !
%%%%%%%%%%%%%%%%%%%%%%%%%%%%%%%%%%%%%%%%%%%%%%%%%%%%%
%
%  pdf layout:
%
\geometry{paperheight=74.25mm}
\usepackage{pgfpages}
\pagestyle{empty}
\pgfpagesuselayout{8 on 1}[a4paper,border shrink=0cm]
\makeatletter
\@tempcnta=1\relax
\loop\ifnum\@tempcnta<9\relax
\pgf@pset{\the\@tempcnta}{bordercode}{\pgfusepath{stroke}}
\advance\@tempcnta by 1\relax
\repeat
\makeatother
% 
%  notes, fields, tags:
%
\newcommand{\xfield}[1]{
        #1\par
        \vfill
        {\tiny\texttt{\parbox[t]{\textwidth}{\localtag\\\globaltag\hfill\uuid}}}
        \newpage}
\newenvironment{field}{}{\newpage}
\newif\ifnote
\newenvironment{note}{\notetrue}{\notefalse}
\newcommand{\localtag}{}
\newcommand{\globaltag}{}
\newcommand{\uuid}{}
\newcommand{\tags}[1]{
    \ifnote 
        \renewcommand{\localtag}{#1}
    \else
        \renewcommand{\globaltag}{#1}
    \fi 
    }
\newcommand{\xplain}[1]{\renewcommand{\uuid}{#1}}
%
%%%%%%%%%%%%%%%%%%%%%%%%%%%%%%%%%%%%%%%%%%%%%%%%%%%%
% The following line again needs to be copied 
% into Anki:
\begin{document}
%%%%%%%%%%%%%%%%%%%%%%%%%%%%%%%%%%%%%%%%%%%%%%%%%%%%

\tags{mathe2::2sem::hilbertraum}

\begin{note}
	\xplain{efb36bac-25bb-4609-b9f4-58c90b79383b}
	\tags{satzt, metrischer-raum}
	
	\begin{field}  % Question
		Definiere ein metrischer Raum (und Metrik).
	\end{field}
	
	\begin{field}  % Answer
		Sien $M \neq \emptyset$ eine Menge und $d: M: M \times M \rightarrow \R{}$ eine Abbildung,
		$d$ hei"st \textit{Metrik} in $M$, wenn $\forall x,y,z \in M$ gilt:
		\begin{enumerate}
			\item (\textbf{Definitheit}) $d(x,y) \geq 0$ und $d(x,y) = 0 \Leftrightarrow x=y $,
			\item (\textbf{Symmetrie}) $d(x,y) = d(y,x)$,
			\item (\textbf{Dreiecksungleichung}) $d(x,y) \leq d(x,z) + d(z,y)$.
		\end{enumerate}
	$(M,d)$ hei"st dann \textit{metrischer Raum}.
	\end{field}
		
	\begin{field}  % Question
		Definiere einen normieren Raum.
	\end{field}
	
	\begin{field}  % Answer
		Seien $V$ Vektorraum ("uber $\R{}$ oder $\C{}$) und $\norm{\cdot} : V \rightarrow \R{}/\C{}$ eine Abbildung. Die Abb. hei"st \textit{Norm} auf $V$, wenn $\forall x,y \in V, \lambda \in \R{} $(bzw. $\in \C{}$) gilt:
		\begin{enumerate}
			\item (\textbf{Definitheit}) $\norm{x} \geq 0$ und $\norm{x} = \Leftrightarrow x =0$,
			\item (\textbf{Homogenit"at}) $\norm{\lambda x} = \abs{\lambda} \norm{x}$,
			\item (\textbf{Dreiecksungleichung}) $\norm{x+y} \leq \norm{x} + \norm{y}$.
		\end{enumerate}
	$(V,\norm{\cdot})$ hei"st \textit{Normierter Raum}.
	\end{field}
		
	\begin{field}  % Question
		Ein \underline{vollst"andiger} normierter Raum hei"st...
	\end{field}
	
	\begin{field}  % Answer
	... \textit{Banach-Raum}.	
	\end{field}
		
	\begin{field}  % Question
		Ein Banach-Raum ist...
	\end{field}
	
	\begin{field}  % Answer
		... ein vollst"andiger normierter Raum.
	\end{field}
	
		
	\begin{field}  % Question
		Wie lautet die Minkowski-Ungleichung?
	\end{field}
	
	\begin{field}  % Answer
		Sei $1\leq p \leq \infty$, $x,y \in l^p$, dann
		\begin{equation*}
			\norm{x + y}_p \leq \norm{x}_p + \norm{y}_p
		\end{equation*}
	
	Also Dreiecksungleichung f"ur die $p$-Norm.
	Analoge Aussage gilt f"ur $L^p$.
	\end{field}
		
	\begin{field}  % Question
		Wie lautets die H"older-Ungleichung?
	\end{field}
	
	\begin{field}  % Answer
		F"ur $x\in l^1, y \in l^\infty$ gilt:
		\begin{equation*}
			xy \coloneqq \qty(x_n y_n)_{n\in \mathbb{N}} \in l^1. \qq{und} \norm{xy}_1 \leq \norm{x}_1 \cdot\norm{y}_\infty
		\end{equation*}
	\end{field}
\end{note}
\begin{note}
	\xplain{72ba8d20-f588-492b-a6d4-6c3a8cfd48e4}
	\tags{definition, skalarprodukt, hilbertraum, 5.2}
	
	\begin{field}  % Question
		Definiere einen unit"aren Raum.
	\end{field}
	
	\begin{field}  % Answer
		Sei $V$ ein Vektorraum "uber $\C{}$. Ein Skalarprodukt auf $V$ ist eine Abbildung $\innerprod{\cdot, \cdot}: V \times V \rightarrow \C{}$ mit den Eigenschaften 
		$\forall \varphi, \psi, \chi \in V, c \in \C{}$ gilt:
		\begin{enumerate}[i)]
			\item $\innerprod{\varphi,\varphi} \geq 0$ und $\innerprod{\varphi,\varphi} = 0 \Leftrightarrow \varphi = 0$,
			\item $\innerprod{\varphi, \psi + \chi} = \innerprod{\varphi,\psi} + \innerprod{\varphi,\chi}$,
			\item $\innerprod{\varphi, c \cdot \psi}= \cdot \innerprod{\varphi, \psi}$,
			\item $\innerprod{\varphi, \psi} = \overline{\innerprod{\varphi,\psi}}$.
		\end{enumerate}
	
	$(V, \innerprod{\cdot, \cdot})$ hei"st \textit{unit"arer Raum}.
	\end{field}

	\begin{field}  % Question
		Ein \textit{Hilbertraum} ist ein...
	\end{field}
	
	\begin{field}  % Answer
		... vollst"andiger unit"arer Raum.
	\end{field}
	
	\begin{field}  % Question
		Ein vollst"andiger unit"arer Raum hei"st...
	\end{field}
	
	\begin{field}  % Answer
		... Hilbertraum. 
	\end{field}
		
	\begin{field}  % Question
		Ein Hilbertraum $\Hi$ hei"st \textit{seperabel}, wenn...
	\end{field}
	\begin{field}  % Answer
		... es eine abz"ahlbare Menge
		$\qty{\varphi_n, n\in \mathbb{N}} \subseteq \Hi$ gibt, die dicht in $\Hi$ liegt.

	\end{field}
\end{note}
	\begin{note}
		\xplain{3046dd82-95de-4b7d-a474-2e2d657331e7}
		\tags{definition, satz, 5.2.9, 5.2.10}
		
		\begin{field}  % Question
			Wie ist ein Abstand von einer Teilmenge in einem Hilbertraum definiert?
		\end{field}
		
		\begin{field}  % Answer
			Seien $\Hi$ ein Hilbertraum, $\phi \in \Hi, K \neq \emptyset, K \subseteq \Hi$ abgeschlossen und konvex. Dann ist:
			\begin{equation*}
				\text{dist}(\phi, K) \coloneqq \inf\qty{\norm{\phi - \psi} | \psi \in K}
			\end{equation*}
		\end{field}
	
		\begin{field}  % Question
			Wie lautet der Satz "uber die Existenz des Abstands?
		\end{field}
		
		\begin{field}  % Answer
			Seine $\Hi$ ein Hilbertraum, $K\subseteq \Hi$ eine abgeschlossene, konvexe Menge und $\varphi_0 \in \Hi$. Dann $\exists! \ \psi_0 \in \Hi$ mit $\norm{\varphi_0 - \psi_0} = \text{dist}(\varphi_0, K)$. Au"serdem gilt:
			\begin{equation*}
				\forall \psi \in K: \real\innerprod{\psi, \varphi_0 - \psi_0} \leq \real\innerprod{\psi_0, \varphi_0 - \psi_0}
			\end{equation*} 
		\end{field}
	\end{note}	
	\begin{note}
		\xplain{ceabc1dc-0f77-450a-8523-3235056376fc}
		\tags{definition, satz, 5.2.11, 5.2.12}
			
		\begin{field}  % Question
			Wie ist das orthogonale Komplement definiert?
		\end{field}
		
		\begin{field}  % Answer
			Seien $\Hi$ ein Hilbertraum und $M \subseteq \Hi$. Die Menge 
			\begin{equation*}
				M^\perp \coloneqq \qty{\varphi \in \Hi| \forall \psi \in M: \varphi \perp \psi}
			\end{equation*} 
			hei"st \textit{orthogonales Komplement} von $M$.
		\end{field}
			
		\begin{field}  % Question
			Wie lautet der Projektionssatz?
		\end{field}
		
		\begin{field}  % Answer
			Seien $\Hi$ ein Hilbertraum, $\L\subset \Hi$ ein abgeschlossener Untervektorraum. Dann hat jedes $\varphi \in \Hi$ eine eindeutige Darstellung:
			\begin{equation*}
				\varphi = \psi + \psi^\perp \qq{mit} \psi \in L \text{ und } \psi^\perp \in L^\perp
			\end{equation*}
			
			$\psi$ hei"st \textit{orthogonale Projektion} von $\varphi$ auf $L$. 
			
			Durch $\varphi \mapsto P_\varphi \coloneqq \psi$ wird ein linearer Operator definiert. Es gilt $P^2=P$ und $\forall \varphi \in \Hi: \norm{P_\varphi} \leq \norm{\varphi}$.
		\end{field}
	%noch direkte Summe.
	\end{note}
	\begin{note}
		\xplain{c58d8f6d-bf9c-4a49-9dcc-5f2f8231b510}
		\tags{satz, ONS, 5.2.13}
		
		\begin{field}  % Question
			Was sind die Aussagen des Satzes "uber die Orthonormalsysteme in Hilbertr"aumen?
		\end{field}
		
		\begin{field}  % Answer
			Seien $\Hi$ ein Hilbertraum und $(\phi_n)$ ein Orthonormalsystem in $\Hi$. Die folgende Aussagen sind "aquivalent:
			\begin{enumerate}[i)]
				\item $\text{span}\qty{\varphi_n | n\in \mathbb{N}}$ ist dicht in $\Hi$.
				\item $\forall \varphi \in \Hi: \ \phi = \sum_{n=1}^\infty \innerprod{\varphi_n, \varphi}\varphi_n \coloneqq \lim\limits_{N\rightarrow \infty}\sum_{n=1}^N \innerprod{\varphi_n, \varphi}\varphi_n$
				\item $\forall \varphi \in \ Hi$ gilt die Parservalsche Gleichung:
					\begin{equation*}
						\norm{\varphi}^2 = \sum_{n=1}^\infty\abs{\innerprod{\varphi_n, \varphi}}^2
					\end{equation*}
			\end{enumerate} 
		\end{field}
		
		\begin{field}  % Question
			Sei $\Hi$ ein Hilbertraum. Wie ist eine \textit{Orthonormalbasis} von $\Hi$ definiert? 
		\end{field}
			
		\begin{field}  % Answer
			Ein Orthonormalsystem $(\phi_n)$ in $\Hi$, f"ur das $\text{span}\qty{\varphi_n \ | \ n\in \mathbb{n}}$ dicht in $\Hi$ ist, hei"st \textit{Orthonormalbasis}. 
		\end{field}
			
		\begin{field}  % Question
			Der Satz "uber die Orthonormalbasis eines Hilbertraumes besagt, dass...
		\end{field}
		
		\begin{field}  % Answer
			jeder Hilbertraum eine Orthonormalbasis hat. 
		\end{field}
			
		\begin{field}  % Question
			Was sind die wichtigen Bemerkungen zum Thema Orthogonalbasen nov Hilbertr"aumen?
		\end{field}
		
		\begin{field}  % Answer
			\begin{itemize}
				\item Reihenfolge von $(\varphi_n)$ ist eigentlich nicht relevant.
				\item $\Hi$ seperabel $\Leftrightarrow$ $\Hi$hat eine anz"ahlbare ONB.
				\item M"achtigkeit allen ONBen zu einem Hilbertraum ist gleich und wird Hilbertraum Dimension von $\Hi$ genannt. 
				\item jeder seperable Hilbertraum ist isomorph zu $l^2$.
			\end{itemize}
		\end{field}
			
		\begin{field}  % Question
			Erkl"are die Isomorphie eines separablen $\Hi$ zu $l^2$.  
		\end{field}
		
		\begin{field}  % Answer
			$(\varphi_n)$ ONB von $\Hi$. Die Abbildung 
			\begin{equation*}
				J: \Hi \rightarrow l^2,\quad \varphi \mapsto \qty(\innerprod{\varphi_n, \varphi})_{n\in \mathbb{N}}
			\end{equation*}
			ist: 
			\begin{itemize}
				\item \textbf{linear} (da das Skalarprodukt im 2. Eintrag es ist. ) ,
				\item \textbf{bijektiv} (da Darstellung bzgl. einer ONB eindeutig ist),
				\item \textbf{isometrisch}.
			\end{itemize}
		Mit \textbf{linear} folgt J \textit{Vektorraumhomomorphismus} weiter mit \textbf{bijektiv} J \textit{Vektorraumisomorphismus}. 
		 mit \textbf{Isometrie} $\Rightarrow$ \textit{isometrische Isomorphie}. 
		\end{field}
	\end{note}
	\begin{note}
		\xplain{51c21e76-0b2c-472a-86c2-845e67a1b33b}
		\tags{definition, satz, dualraum, 5.2.17, 5.2.18}
		
		\begin{field}  % Question
			Definiere den \textit{Dualraum} zu $\Hi$.
		\end{field}
		
		\begin{field}  % Answer
			Der Dualraum zu $\Hi$ ist der Vektorraum aller linearen stetigen Funktionale auf $\Hi$
			mit der Norm:
			\begin{equation*}
				\norm{f}_{\Hi'} \coloneqq \sup\qty{\abs{f(\varphi)} \ | \ \varphi \in \Hi, \norm{\varphi}\leq 1}
			\end{equation*}
			Er wird mit $\Hi'$ bezeichnet. 
		\end{field}
		
	\begin{field}  % Question
		Wie lautet der Satz von Riesz?
	\end{field}
	
	\begin{field}  % Answer
		\begin{itemize}
			\item $\forall f \in \Hi'$ gibt es ein \textbf{eindeutiges} $\psi \in \Hi$ so, dass
			$f \coloneqq f_\psi$, wobei $f_\psi :\Hi \rightarrow \C{}, \ \ \varphi \mapsto \innerprod{\psi,\varphi}$.
			\item Die Abbildung $J : \Hi \rightarrow \Hi', \ \ \psi \mapsto f_\psi$ ist \textbf{bijektiv} und \textbf{konjugiert linear}.
			\item Es gilt $\forall \psi \in \Hi: \quad \norm{\Psi} = \norm{f_\Psi}_{\Hi'}$.
		\end{itemize}
	\end{field}
	\end{note}

\end{document}
