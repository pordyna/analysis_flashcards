% !TEX encoding = UTF-8 Unicode
%%%%%%%%%%%%%%%%%%%%%%%%%%%%%%%%%%%%%%%%%%%%%%%%%%%%
% The first part of the header needs to be copied
%       into the note options in Anki.
%%%%%%%%%%%%%%%%%%%%%%%%%%%%%%%%%%%%%%%%%%%%%%%%%%%%

% layout in Anki:
\documentclass[9pt]{article}
\usepackage[a4paper]{geometry}
\geometry{paperwidth=.5\paperwidth,paperheight=100in,left=2em,right=2em,bottom=1em,top=2em}
\pagestyle{empty}
\setlength{\parindent}{0in}
 
% hyphenation:
\usepackage[ngerman]{babel}

% encoding:
\usepackage[T1]{fontenc}
\usepackage[utf8]{inputenc}
\usepackage{lmodern}

% packages:
\usepackage{parskip}
\usepackage[free-standing-units=true]{siunitx}
% Writng si units, numbers, list of numbers etc.
\usepackage{gensymb}
% Unified typseting of units outside of siuitx
\usepackage{amsmath}
\usepackage{amsfonts}  
% Math features
\usepackage{esdiff}
% Typeseting of (partial)derivatives 
\usepackage{commath}
% More  derivatives. Not so nice like esdiif but adds
% \dif comand for upright d in math mode.
\usepackage{bm}
% Bold font in math mode
\usepackage{esint}
% Some fancy integrals signs. Mutilpe integrals
\usepackage{enumerate}
% Different styles for enumerate lists
\usepackage{multirow}
% More advanced tabular
\usepackage{physics}
% Lots of usefull comannds for physicists. Vektors, nablas etc.
\usepackage{amssymb, xfrac}
% extra fonts and symbols 
\usepackage{mathtools}
% extension to amsmath, fixes, meany new tool
\usepackage{isotope}

%commands:
\usepackage{ifthen}

\newcommand*{\tran}{^{\mkern-1.5mu\mathsf{T}}}
\newcommand{\Rn}{\mathbb{R}^n}
\newcommand{\Rk}{\mathbb{R}^k}
\newcommand{\R}[1]{%
	\ifthenelse{\equal{#1}{}}
		{\mathbb{R}}
		{\mathbb{R}^#1}}%
\newcommand{\C}[1]{%
	\ifthenelse{\equal{#1}{}}
	{\mathbb{C}}
	{\mathbb{C}^#1}}%	
\renewcommand{\vec}[1]{\underline{#1}}

%%%%%%%%%%%%%%%%%%%%%%%%%%%%%%%%%%%%%%%%%%%%%%%%%%%%
% Following part of header NOT to be copied into
%            the note options in Anki.
%          ! Anki will throw an errow !
%%%%%%%%%%%%%%%%%%%%%%%%%%%%%%%%%%%%%%%%%%%%%%%%%%%%%
%
%  pdf layout:
%
\geometry{paperheight=74.25mm}
\usepackage{pgfpages}
\pagestyle{empty}
\pgfpagesuselayout{8 on 1}[a4paper,border shrink=0cm]
\makeatletter
\@tempcnta=1\relax
\loop\ifnum\@tempcnta<9\relax
\pgf@pset{\the\@tempcnta}{bordercode}{\pgfusepath{stroke}}
\advance\@tempcnta by 1\relax
\repeat
\makeatother
% 
%  notes, fields, tags:
%
\newcommand{\xfield}[1]{
        #1\par
        \vfill
        {\tiny\texttt{\parbox[t]{\textwidth}{\localtag\\\globaltag\hfill\uuid}}}
        \newpage}
\newenvironment{field}{}{\newpage}
\newif\ifnote
\newenvironment{note}{\notetrue}{\notefalse}
\newcommand{\localtag}{}
\newcommand{\globaltag}{}
\newcommand{\uuid}{}
\newcommand{\tags}[1]{
    \ifnote 
        \renewcommand{\localtag}{#1}
    \else
        \renewcommand{\globaltag}{#1}
    \fi 
    }
\newcommand{\xplain}[1]{\renewcommand{\uuid}{#1}}
%
%%%%%%%%%%%%%%%%%%%%%%%%%%%%%%%%%%%%%%%%%%%%%%%%%%%%
% The following line again needs to be copied 
% into Anki:
\begin{document}
%%%%%%%%%%%%%%%%%%%%%%%%%%%%%%%%%%%%%%%%%%%%%%%%%%%%

\tags{mathe2::1sem::GDGL}

\begin{note}
	\xplain{UUID}  %UUID
	\tags{definition, 3.1.1, 3.1.2}
	
	\begin{field}  % Question
		Definiere  eine gew"ohnliche Differentialgleichung (implizit).
		Was versteht man unter einer L"osung der DGL?
	\end{field}  
	
	\begin{field}  % Answer
		Seien G ein Gebiet im $\R{n+2}$, $I$ ein Intervall, eine Funktion $x: I \rightarrow \R{}, 
		t\rightarrowtail x(t)$ $n$ mal differenzierbar und $F: G \rightarrow \R{}$.
		Dann hei"st
		\begin{equation*}
			F(t,x,\dot{x}, \dots, x^{(n)}) = 0
		\end{equation*}
		
		\textit{implizite} gew"ohnliche DGL der Ordnung $n$. 
		
		Sei $x\in C^n$, $x: (a,b) \rightarrowtail \R{}$.  $x$ ist eine L"osung der DGL falls:
		\begin{enumerate}
			\item $\qty(t,x(t), \dot{x}(t), \dots x^{(n)}(t)) \in G \qquad \forall_{t\in (a,b)}$, und
			\item die Gleichung $F=0$ ist erf"uhlt $\forall_{t\in (a,b)}$.
		\end{enumerate}
	\end{field}
	
	\begin{field}  % Question
		Definiere  eine gew"ohnliche Differentialgleichung (explizit).
	\end{field}  
	
	\begin{field}  % Answer
				Sei $\tilde{G}$ ein Gebiet im $\R{n+1}$,  $I$ ein Intervall, eine Funktion $x: I \rightarrow \R, 
		t\rightarrowtail x(t)$ $n$ mal differenzierbar und $f: \tilde{G} \rightarrow \R{}.$
		Dann hei"st
		\begin{equation*}
			x^{(n)} = f(t,x,\dot{x}, \dots, x^{(n-1)})
		\end{equation*}
		\textit{explizite}  gew"ohnliche DGL der Ordnung $n$.
		
		Sei $x\in C^n$, $x: (a,b) \rightarrowtail \R{}$.  $x$ ist eine L"osung der DGL falls:
		\begin{enumerate}
			\item $\qty(t,x(t), \dot{x}(t), \dots x^{(n-1)}(t)) \in \tilde{G} \qquad \forall_{t\in (a,b)}$, und
			\item die Gleichung $f=0$ ist erf"uhlt f"ur alle $t\in (a,b)$.
		\end{enumerate}
	\end{field}
\end{note}

\begin{note}
	\xplain{UUID}  %UUID
	\tags{definition, 3.1.3}
	
	\begin{field}  % Question
		Ein Anfangswertproblem  hei"st \textit{korrekt gestellt}, wenn..
	\end{field}  
	
	\begin{field}  % Answer
		genau eine L"osung existiert und eine stetige Abhängigkeit von den Anfangsbedingungen gew"ahrleistet ist. 
	\end{field}
\end{note}

\begin{note}
	\xplain{UUID}  %UUID
	\tags{GDGL, methode, 3.2.7.1}
	
	\begin{field}  % Question
		Wie l"ost man $\dot{x} + f(t)x = g(t)$ mit der Eulerschen Methode? 
	\end{field}  
	
	\begin{field}  % Answer
		\begin{itemize}
			\item Multipliziere mit $\exp(\int_{t_0}^{t}f(t') \dif t')$.
			\item Fasse LHS als eine Abletiung nach $x$.
			\item Integriere es auf.
		\end{itemize}
	\end{field}
\end{note}

\begin{note}
	\xplain{UUID}  %UUID
	\tags{satz, 3.3 , fixpunkt}
	
	\begin{field}  % Question
		Wie lautet der Banachsche Fixpunktsatz?
	\end{field}  
	
	\begin{field}  % Answer
		Sei $(X,d)$ vollst"andiger metrischer Raum, sei $A \subseteq X$ abgeschlossen,
		$T: A \rightarrow A$ kontrahierend mit Konktraktionszahl $q$. Dann:
		\begin{enumerate}
			\item $T$ hat genau einen Fixpunkt $x^*$ in $A$,
			\item f"ur beliebige $x_0 \in A$ konvergiert $x_{n+1} = T{x_n}$ gegen $x^*$ mit $n \in \mathbb{N}$,
			\item es gilt die Absch"atzung:
			\begin{equation*}
				\mathrm{d} (x_n,x^*) \leq \frac{q^n}{1-q}\mathrm{d}(x_1,x_0)
			\end{equation*} 
		\end{enumerate}
	\end{field}
\end{note}

\begin{note}
	\xplain{UUID}  %UUID
	\tags{definition, Lippschitz, 3.4.1, 3.4.8}
	
	\begin{field}  % Question
		 $f:G\subseteq \R{2} \rightarrow \R{} \quad (t,x) \rightarrowtail f(t,x)$ gen"ugt einer
		 \textit{Lippschitzbedingung} bzg. des 2. Arguments auf $G$, wenn
	\end{field}  
	
	\begin{field}  % Answer
		$\exists L>0 \quad \forall t, x_1, x_2 \qq{mit} (t,x_1), (t,x_2) \in G$
		\begin{equation*}
			 \abs{f((t,x1) - f(t,x_2))} \leq L\abs{x_1 -x_2}
		\end{equation*}
	\end{field}
		
	\begin{field}  % Question
		Definiere die Lippschitzbedingung f"ur Vektorfunktionen.
	\end{field}
	
	\begin{field}  % Answer
		$\vec{f}: \R{n+1} \supseteq D(\vec{f} \rightarrow \R{n} :\  (t,\vec{x}) \rightarrowtail f(t, vec{x})$
		gen"ugt einer \textit{Lippschitzbedingung} bzgl. $\vec{x}$ in $D(\vec{f})$, wenn
		$\forall \vec{x}, \vec{y}$ mit $(t,\vec{x}), (t,\vec{y}) \in D(\vec{f}) \qq \exists L >0$:
		\begin{equation*}
			\norm{\vec{f}(t,\vec{x}) - \vec{f}(t, \vec{y})}_n \leq L \norm{\vec{x} - \vec{y}}_n
		\end{equation*}
		
		$\norm{\cdot}_n$ ist beliebige Norm in $\Rn$.
	\end{field}
\end{note}
\begin{note}
	\xplain{UUID}
	\tags{satz, picard-lindeloef, 3.4.2}
	
	\begin{field}  % Question
		Wie lautet der Satz von Picard-Lindel"of "uber die Existenz und Eindeutigkeit der L"osung.
	\end{field}
	
	\begin{field}  % Answer
		Sei $\dot{x} = f(t,x)$ mit $x_0= x(t_0)$ ein Anfangswertproblem (AWP) gegeben, $f$ erf"ulle die folgenden Bedingungen:
		\begin{itemize}
			\item $\exists a,b \in \R{}_{>0}$ so, dass $f$ auf dem Rechteck
			\begin{equation*}
				Q\coloneqq \qty{(t,x) \in \R{2}: \abs{t-t_0} \leq a, \abs{x-x_0} \leq b}
			\end{equation*}
			stetig und durch $M$ beschr"ankt ist. 
			\item $f$ ist auf $Q$ Lippschitzstetig bzg. $x$ mit Lippschitzkonstante $L$. 
		\end{itemize}
		Dann existiert geanu eine lokale L"osung des AWP, d.h. $\exists \sigma>0$ so ,dass auf
			$J\coloneqq \qty[t_0 - \sigma, t_0 + \sigma]$  genau eine L"osung existiert. Man kann $\sigma$ so w"ahlen: $\sigma < \min\qty{a, \frac{b}{m}, \frac{1}{L}}$.
	\end{field}
\end{note}

\begin{note}
	\xplain{UUID}
	\tags{definition, satz, DGL-System}
	
	\begin{field}  % Question
		
	\end{field}
	
	\begin{field}  % Answer
		
	\end{field}
\end{note}

\end{document}