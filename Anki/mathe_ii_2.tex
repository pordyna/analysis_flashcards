% !TeX spellcheck = de_DE
% !TEX encoding = UTF-8 Unicode
%%%%%%%%%%%%%%%%%%%%%%%%%%%%%%%%%%%%%%%%%%%%%%%%%%%%
% The first part of the header needs to be copied
%       into the note options in Anki.
%%%%%%%%%%%%%%%%%%%%%%%%%%%%%%%%%%%%%%%%%%%%%%%%%%%%

% layout in Anki:
\documentclass[9pt]{article}
\usepackage[a4paper]{geometry}
\geometry{paperwidth=.5\paperwidth,paperheight=100in,left=2em,right=2em,bottom=1em,top=2em}
\pagestyle{empty}
\setlength{\parindent}{0in}
 
% hyphenation:
\usepackage[ngerman]{babel}

% encoding:
\usepackage[T1]{fontenc}
\usepackage[utf8]{inputenc}
\usepackage{lmodern}

% packages:
\usepackage{parskip}
\usepackage[free-standing-units=true]{siunitx}
% Writng si units, numbers, list of numbers etc.
\usepackage{gensymb}
% Unified typseting of units outside of siuitx
\usepackage{amsmath}
\usepackage{amsfonts}  
% Math features
\usepackage{esdiff}
% Typeseting of (partial)derivatives 
\usepackage{commath}
% More  derivatives. Not so nice like esdiif but adds
% \dif comand for upright d in math mode.
\usepackage{bm}
% Bold font in math mode
\usepackage{esint}
% Some fancy integrals signs. Mutilpe integrals
\usepackage{enumerate}
% Different styles for enumerate lists
\usepackage{multirow}
% More advanced tabular
\usepackage{physics}
% Lots of usefull comannds for physicists. Vektors, nablas etc.
\usepackage{amssymb, xfrac, bbold}
% extra fonts and symbols 
\usepackage{mathtools}
% extension to amsmath, fixes, meany new tool
\usepackage{isotope}
\usepackage{empheq}
%commands:
\usepackage{ifthen}

\newcommand*{\tran}{^{\mkern-1.5mu\mathsf{T}}}
\newcommand{\Rn}{\mathbb{R}^n}
\newcommand{\Rk}{\mathbb{R}^k}
\newcommand{\R}[1]{%
	\ifthenelse{\equal{#1}{}}
		{\mathbb{R}}
		{\mathbb{R}^{#1}}}%
\newcommand{\C}[1]{%
	\ifthenelse{\equal{#1}{}}
	{\mathbb{C}}
	{\mathbb{C}^{#1}}}%	
\renewcommand{\vec}[1]{\underline{#1}}

%%%%%%%%%%%%%%%%%%%%%%%%%%%%%%%%%%%%%%%%%%%%%%%%%%%%
% Following part of header NOT to be copied into
%            the note options in Anki.
%          ! Anki will throw an errow !
%%%%%%%%%%%%%%%%%%%%%%%%%%%%%%%%%%%%%%%%%%%%%%%%%%%%%
%
%  pdf layout:
%
\geometry{paperheight=74.25mm}
\usepackage{pgfpages}
\pagestyle{empty}
\pgfpagesuselayout{8 on 1}[a4paper,border shrink=0cm]
\makeatletter
\@tempcnta=1\relax
\loop\ifnum\@tempcnta<9\relax
\pgf@pset{\the\@tempcnta}{bordercode}{\pgfusepath{stroke}}
\advance\@tempcnta by 1\relax
\repeat
\makeatother
% 
%  notes, fields, tags:
%
\newcommand{\xfield}[1]{
        #1\par
        \vfill
        {\tiny\texttt{\parbox[t]{\textwidth}{\localtag\\\globaltag\hfill\uuid}}}
        \newpage}
\newenvironment{field}{}{\newpage}
\newif\ifnote
\newenvironment{note}{\notetrue}{\notefalse}
\newcommand{\localtag}{}
\newcommand{\globaltag}{}
\newcommand{\uuid}{}
\newcommand{\tags}[1]{
    \ifnote 
        \renewcommand{\localtag}{#1}
    \else
        \renewcommand{\globaltag}{#1}
    \fi 
    }
\newcommand{\xplain}[1]{\renewcommand{\uuid}{#1}}
%
%%%%%%%%%%%%%%%%%%%%%%%%%%%%%%%%%%%%%%%%%%%%%%%%%%%%
% The following line again needs to be copied 
% into Anki:
\begin{document}
%%%%%%%%%%%%%%%%%%%%%%%%%%%%%%%%%%%%%%%%%%%%%%%%%%%%

\tags{mathe2::1sem::GDGL}

\begin{note}
	\xplain{UUID}  %UUID
	\tags{definition, 3.1.1, 3.1.2}
	
	\begin{field}  % Question
		Definiere  eine gew"ohnliche Differentialgleichung (implizit).
		Was versteht man unter einer L"osung der DGL?
	\end{field}  
	
	\begin{field}  % Answer
		Seien G ein Gebiet im $\R{n+2}$, $I$ ein Intervall, eine Funktion $x: I \rightarrow \R{}, 
		t\mapsto x(t)$ $n$ mal differenzierbar und $F: G \rightarrow \R{}$.
		Dann hei"st
		\begin{equation*}
			F(t,x,\dot{x}, \dots, x^{(n)}) = 0
		\end{equation*}
		
		\textit{implizite} gew"ohnliche DGL der Ordnung $n$. 
		
		Sei $x\in C^n$, $x: (a,b) \mapsto \R{}$.  $x$ ist eine L"osung der DGL falls:
		\begin{enumerate}
			\item $\qty(t,x(t), \dot{x}(t), \dots x^{(n)}(t)) \in G \qquad \forall_{t\in (a,b)}$, und
			\item die Gleichung $F=0$ ist erf"uhlt $\forall_{t\in (a,b)}$.
		\end{enumerate}
	\end{field}
	
	\begin{field}  % Question
		Definiere  eine gew"ohnliche Differentialgleichung (explizit).
	\end{field}  
	
	\begin{field}  % Answer
				Sei $\tilde{G}$ ein Gebiet im $\R{n+1}$,  $I$ ein Intervall, eine Funktion $x: I \rightarrow \R, 
		t\mapsto x(t)$ $n$ mal differenzierbar und $f: \tilde{G} \rightarrow \R{}.$
		Dann hei"st
		\begin{equation*}
			x^{(n)} = f(t,x,\dot{x}, \dots, x^{(n-1)})
		\end{equation*}
		\textit{explizite}  gew"ohnliche DGL der Ordnung $n$.
		
		Sei $x\in C^n$, $x: (a,b) \mapsto \R{}$.  $x$ ist eine L"osung der DGL falls:
		\begin{enumerate}
			\item $\qty(t,x(t), \dot{x}(t), \dots x^{(n-1)}(t)) \in \tilde{G} \qquad \forall_{t\in (a,b)}$, und
			\item die Gleichung $f=0$ ist erf"uhlt f"ur alle $t\in (a,b)$.
		\end{enumerate}
	\end{field}
\end{note}

\begin{note}
	\xplain{UUID}  %UUID
	\tags{definition, 3.1.3}
	
	\begin{field}  % Question
		Ein Anfangswertproblem  hei"st \textit{korrekt gestellt}, wenn..
	\end{field}  
	
	\begin{field}  % Answer
		genau eine L"osung existiert und eine stetige Abhängigkeit von den Anfangsbedingungen gew"ahrleistet ist. 
	\end{field}
\end{note}

\begin{note}
	\xplain{UUID}  %UUID
	\tags{GDGL, methode, 3.2.7.1}
	
	\begin{field}  % Question
		Wie l"ost man $\dot{x} + f(t)x = g(t)$ mit der Eulerschen Methode? 
	\end{field}  
	
	\begin{field}  % Answer
		\begin{itemize}
			\item Multipliziere mit $\exp(\int_{t_0}^{t}f(t') \dif t')$.
			\item Fasse LHS als eine Abletiung nach $x$.
			\item Integriere es auf.
		\end{itemize}
	\end{field}
\end{note}

\begin{note}
	\xplain{UUID}  %UUID
	\tags{satz, 3.3 , fixpunkt}
	
	\begin{field}  % Question
		Wie lautet der Banachsche Fixpunktsatz?
	\end{field}  
	
	\begin{field}  % Answer
		Sei $(X,d)$ vollst"andiger metrischer Raum, sei $A \subseteq X$ abgeschlossen,
		$T: A \rightarrow A$ kontrahierend mit Konktraktionszahl $q$. Dann:
		\begin{enumerate}
			\item $T$ hat genau einen Fixpunkt $x^*$ in $A$,
			\item f"ur beliebige $x_0 \in A$ konvergiert $x_{n+1} = T{x_n}$ gegen $x^*$ mit $n \in \mathbb{N}$,
			\item es gilt die Absch"atzung:
			\begin{equation*}
				\mathrm{d} (x_n,x^*) \leq \frac{q^n}{1-q}\mathrm{d}(x_1,x_0)
			\end{equation*} 
		\end{enumerate}
	\end{field}
\end{note}

\begin{note}
	\xplain{UUID}  %UUID
	\tags{definition, Lippschitz, 3.4.1, 3.4.8}
	
	\begin{field}  % Question
		 $f:G\subseteq \R{2} \rightarrow \R{} \quad (t,x) \mapsto f(t,x)$ gen"ugt einer
		 \textit{Lippschitzbedingung} bzg. des 2. Arguments auf $G$, wenn
	\end{field}  
	
	\begin{field}  % Answer
		$\exists L>0 \quad \forall t, x_1, x_2 \qq{mit} (t,x_1), (t,x_2) \in G$
		\begin{equation*}
			 \abs{f((t,x1) - f(t,x_2))} \leq L\abs{x_1 -x_2}
		\end{equation*}
	\end{field}
		
	\begin{field}  % Question
		Definiere die Lippschitzbedingung f"ur Vektorfunktionen.
	\end{field}
	
	\begin{field}  % Answer
		$\vec{f}: \R{n+1} \supseteq D(\vec{f} \rightarrow \R{n} :\  (t,\vec{x}) \mapsto f(t, vec{x})$
		gen"ugt einer \textit{Lippschitzbedingung} bzgl. $\vec{x}$ in $D(\vec{f})$, wenn
		$\forall \vec{x}, \vec{y}$ mit $(t,\vec{x}), (t,\vec{y}) \in D(\vec{f}) \qq \exists L >0$:
		\begin{equation*}
			\norm{\vec{f}(t,\vec{x}) - \vec{f}(t, \vec{y})}_n \leq L \norm{\vec{x} - \vec{y}}_n
		\end{equation*}
		
		$\norm{\cdot}_n$ ist beliebige Norm in $\Rn$.
	\end{field}
\end{note}
\begin{note}
	\xplain{UUID}
	\tags{satz, picard-lindeloef, 3.4.2}
	
	\begin{field}  % Question
		Wie lautet der Satz von Picard-Lindel"of "uber die Existenz und Eindeutigkeit der L"osung.
	\end{field}
	
	\begin{field}  % Answer
		Sei $\dot{x} = f(t,x)$ mit $x_0= x(t_0)$ ein Anfangswertproblem (AWP) gegeben, $f$ erf"ulle die folgenden Bedingungen:
		\begin{itemize}
			\item $\exists a,b \in \R{}_{>0}$ so, dass $f$ auf dem Rechteck
			\begin{equation*}
				Q\coloneqq \qty{(t,x) \in \R{2}: \abs{t-t_0} \leq a, \abs{x-x_0} \leq b}
			\end{equation*}
			stetig und durch $M$ beschr"ankt ist. 
			\item $f$ ist auf $Q$ Lippschitzstetig bzg. $x$ mit Lippschitzkonstante $L$. 
		\end{itemize}
		Dann existiert geanu eine lokale L"osung des AWP, d.h. $\exists \sigma>0$ so ,dass auf
			$J\coloneqq \qty[t_0 - \sigma, t_0 + \sigma]$  genau eine L"osung existiert. Man kann $\sigma$ so w"ahlen: $\sigma < \min\qty{a, \frac{b}{m}, \frac{1}{L}}$.
	\end{field}
\end{note}

\begin{note}
	\xplain{UUID}
	\tags{definition, satz, DGL-System}
	
	\begin{field}  % Question
		Ein explizites Differentialgleichungssystem $n$-ter Ordnung der Dimension $k$ ist definiert als: 
	\end{field}
	
	\begin{field}  % Answer
		\begin{equation*}
			\vec{x}^{(n)} = \vec{f}(t, \vec{x}(t), \dots, \vec{x}^{n-1}(t))
		\end{equation*}
		wobei $\vec{x}(t) = \begin{pmatrix} x_1(t)\\ \vdots\\ x_k(t)  \end{pmatrix}$,
		$f: \R{} \times \R{}^{n\cdot k} \supseteq D(f) \rightarrow \R{k}$. 
	\end{field}
		
	\begin{field}  % Question
		Wie l"asst sich eine Differentialgleichung $n$-ter Ordnung auf ein Differentialgleichungssystem 1. Ordnung der Dimension $k$ transformieren?
	\end{field}
	
	\begin{field}  % Answer
		Sei $x  \in C^n((\alpha, \beta), \R{})$ eine L"osung einer skalaren DGL $n$-ter Ordnung
		($x^{(n)} = f(t,x, \dots, x^{(n-1)})$ eventuell mit Anfangsbedingungen. 
		 
		Definiere $\vec{z}(t) \in \R{n}$ mit $ z_i(t) = x(t)^{(i-1)}$,
		  $i\in \qty{1,\dots,n}$. Es gilt $\dot{z}_i = z_{i+1} = x^{(i)}$.
		
		$z = \vec{z}(t)$ eine L"osung des $n$-dim. DGL-Systems 1. Ordnung:
		\footnotesize
		\begin{equation*}
			\vec{\dot{z}}(t) = \begin{pmatrix}z_2(t) \\ \vdots \\ z_n(t) \\ f(t, \vec{z})  \end{pmatrix} = \vec{g}(t,\vec{z}) \qquad D(\vec{g}) = (\alpha,\beta) \times \Rn
		\end{equation*} 
		\normalsize
		F"ur $\vec{z}(t_0) = \vec{z_0}$  setze  $z^0_i = x^{(i-1)}(t_0)$. 
	\end{field}
\end{note}

\begin{note}
	\xplain{UUID}
	\tags{definition, fortsetzung, 3.5.1}
	
	\begin{field}  % Question
		Definiere den Begriff einer Fortsetzung einer L"osung. 
	\end{field}
	
	\begin{field}  % Answer
		Eine L"osung $y$ des AWP 
		\begin{equation*}
		\dot{x}=f(t,x(t)), D(f) \subseteq \R{} \times \R, x(t_0)=X_0
		\end{equation*}
		auf einem Intervall $(a',b')$ hei"st \textit{Fortsetzung von x} (x eine lokale L"osung des AWP auf $(a,b)$), wenn:
		\begin{itemize}
			\item $(a,b) \subset (a',b')$,
			\item $y(t) \equiv x(t) \forall_{t\in (a,b)}$
		\end{itemize}
	\end{field}
\end{note}
\begin{note}
	\xplain{UUID}
	\tags{satz, fortsetzung, 3.5.2}
	
	\begin{field}  % Question
		Wie lautet der Satz "uber die Eindeutigkeit der Fortsetzung?
	\end{field}
		
	\begin{field}  % Answer
		Sei ein  
		\begin{empheq}[left= \text{AWP:~} \empheqlbrace]{align*}
		&\dot{x}=f(t,x(t)), D(f) \subseteq \R{} \times \R{} \\ 
		 &x(t_0)=x_0
		\end{empheq}
		gegeben.
		
		 Sei $Q=[t_1,t_2] \times [x_1,x_2]$ eine Menge auf der das AWP mit $(t_0, x_0) \in Q$
		lokal l"osbar ist. Sei $x$ auf $(a,b) \in [t_1,t_2]$ eine L"osung des AWPs. Seien $y_1$, $y_2$
		zwei Fortsetzungen von x auf $(a',b')\in [t_1,t_2]$. Dann gilt: 
		\begin{equation*}
		y_1(t) = y_2(t) \qquad \forall_{t\in (a',b')}
		\end{equation*}
	\end{field}
\end{note}
\begin{note}
	\xplain{UUID}
	\tags{definition, fortsetzung,, 3.5.3, 3.5.4}
	
	\begin{field}  % Question
		Eine L"osung, die nicht mehr fortsetzbar ist, hei"st...
	\end{field}
	
	\begin{field}  % Answer
		... \textit{maximal}.
	\end{field}
		
	\begin{field}  % Question
		Eine L"osung hei"st \textit{maximal}, wenn...
	\end{field}
	
	\begin{field}  % Answer
		... sie nicht mehr fortsetzbar ist.
	\end{field}
		
	\begin{field}  % Question
		Wie lautet der Satz "uber die maximale L"osung?
	\end{field}
	
	\begin{field}  % Answer
		Sei $G \subset \R{2}$ ein beschr"anktes Gebiet. $f: G \rightarrow \R{}$ gen"uge den Bedingungen vom Satz von Picard/Lindel"of. Dann gilt:
		\begin{enumerate}
			\item $\exists!$ eine maximal L"osung $x_{\text{max}}$ des AWP (auf $(a,b)$).
			\item F"ur $u \coloneqq \lim\limits_{t\rightarrow a^+}x_{\text{max}}(t)$,
			$v\coloneqq \lim\limits_{t\rightarrow b^-}x_{\text{max}}(t)$ \\
			gilt $(a,u), (b,v) \in \partial G$.
		\end{enumerate}
	\end{field}
\end{note}
\begin{note}
	\xplain{UUID}
	\tags{satz, 3.6}
	
	\begin{field}  % Question
		Wie lautet der Satz "uber die Absch"atzung der Differenz von L"osungen (stetige Abhängigkeit)?
	\end{field}
	
	\begin{field}  % Answer
		Sei $\dot{x}=f(t,x)$, $f$ stetig auf einem Streifen $(a,b)\times \R{}$. F"ur jedes abgeschlossene Intervall $[a',b'] \subset (a,b)$ existieren eine Lippschitzkonstante $L'$ mit:
		\begin{equation*}
			\forall_{t \in [a', b']} \forall_{x_1, x_2 \in \R{}}: \quad
			\abs{f(t,x_1) - f(t,x_2)} \leq L'\norm{x_1 - x_2}
		\end{equation*}
		
		Seien nun $x(t), \hat{x}(t)$ L"osungen eines AWP mit $x(t_0)=x_0, \\  \hat{x}(t_0) = \hat{x}_0$
		auf $[a',b']$ MIT $t_o \in (a',b')$. Dann gilt:
		\begin{equation*}
			\forall_{t \in [a', b']}: \ \abs{x(t) - \hat{x}(t)} \leq e^{L'\abs{t-t_0}} \cdot \norm{x_0-\hat{x}_0}
		\end{equation*}
	\end{field}
\end{note}

\begin{note}
	\xplain{UUID}
	\tags{definition, satz LDGL, 3.7}
	
	\begin{field}  % Question
		Wie ist ein linearer DGL-System 1. Ordnung definiert?
	\end{field}
	
	\begin{field}  % Answer
		\begin{equation*}
		\dot{x} = A(t) f(t), \qq{homogen falls} f(t) =0
		\end{equation*}
		\footnotesize
		$x(t) = \begin{pmatrix}x_1(t) \\ \vdots \\ x_n(t) \end{pmatrix}$, \quad $A(t)=
			\begin{pmatrix}
				a_{11}(t) & \dots & a_{1n}(t) \\
				\vdots    &       & \vdots \\
				a_{n1}(t) & \dots & a_{nn}(t) 
			\end{pmatrix}$,
		
		
		$f(t) = 
			\begin{pmatrix}
				f_1(t) \\ \vdots \\ f_n(t)
			\end{pmatrix}$
		gegeben auf $t\in I = (a,b)$
	\end{field}
		
	\begin{field}  % Question
		Gibt den Satz "uber die
		Existenz und Eindeutigkeit der L"osung eines AWP zu einer linearen DGL an.
	\end{field}
	
	\begin{field}  % Answer
		Sei $L(t,x)=f(t)$, $x^{(i)}(t_0) = \chi_i$, $i \in \qty{0,\dots, n-1}$ ein AWP.
		
		Die Koeffizienten $a_i(t)$ von $L(t,x)$ seien aus $\C{}$ oder $\R{}$ und stetig auf $I \subset \R{}$, $f$ stetig auf $I$. Seien $t_0\in I, \chi_i \in \R{}$  gegeben.
		
		 Dann besitzt das AWP genau eine L"osung.
		 
		 Diese existiert auf dem ganzen Intervall $I$ und h"angt auf jedem kompakten Teilintervall von $I$ von den ABen $a_i(t),f(t)$ stetig ab.
	\end{field}
\end{note}
\begin{note}
	\xplain{UUID}
	\tags{LDGLS}
	
	\begin{field}  % Question
		Sei ein linearer homogener DGL-System 1. Ordnung und $n$-ten Dimension gegeben. Dann hei"st ein System von 
		$n$ lin. unabh"angigen L"osungen...
	\end{field}
	
	\begin{field}  % Answer
		\textit{Fundamentalsystem.}
	\end{field}
		
	\begin{field}  % Question
		Die Fundametalmatrix ist...
	\end{field}
	
	\begin{field}  % Answer
		die L"osungsmatrix $\Phi(t)$ ($(x_1|x_2|...)$ L"osungen als Spalten.) aus $n$ lin. unabh"angigen L"osungen. (Das zugeh"orige homogene LDGLS ist $n$-ter Ordnung).
	\end{field}
		
	\begin{field}  % Question
		Gib drei äquivalente Aussage zu: Eine quadratische L"osungsmatrix $\Phi(t)$ ist die Fundamentalmatrix.
	\end{field}
	
	\begin{field}  % Answer
		\begin{itemize}
			\item Die Spalten von $\Phi(t)$ bzw. die einzelne L"osungen sind linear unabh"angig ,
			\item $\forall_{t\in I}: \quad \text{rang}\Phi(t)$ ist maximal,
			\item $\exists_{t\in I}: \quad \text{rang}\Phi(t)$ ist maximal.
		\end{itemize}
	$I$ Stetigkeitsintervall auf dem das LDGLS gegeben ist.  
	\end{field}
		
	\begin{field}  % Question
		Wronskideterminante ist definiert als...
	\end{field}
	
	\begin{field}  % Answer
		... die Determinante der Fundamentalmatrix.
	\end{field}
		
	\begin{field}  % Question
		Die Determinante der Fundamentalmatrix hei"st...
	\end{field}
	
	\begin{field}  % Answer
		... Wronskideterminante.
	\end{field}
		
	\begin{field}  % Question
		Satz "uber die Wronski-Determinante.
		F"ur $n$ L"osungen eines homogenen LDGLS 1.Ordnung und $n$-ter Dimension sind äquivalent: 
		\begin{itemize}
			\item Die L"osungen bilden ein Fundamentalsystem.
			\item ...
		\end{itemize}  
	\end{field}
	
	\begin{field}  % Answer
		...
		\begin{itemize}
			\item $\forall t \in I$ $W(t)\neq0$,
			\item $\exists t \in I$ $W(t)\neq0$,
		\end{itemize}
	$I$ Stetigkeitsintervall auf dem das LDGLS gegeben ist.
	$W(t)$ Wronski-Determinante.
	\end{field}
		
	\begin{field}  % Question
		Satz "uber die Existenz und Eindeutigkeit der Fundamentalmatrix lautet:
		
		(+Beweisidee)
	\end{field}

	\begin{field}  % Answer
		Es g"abe ein homogenes lineares DGL-System mit der Ordnung 1 und Dimension $n$.
		\begin{equation*}
			\dot{\vec{x}} = A(t)\vec{x} 
		\end{equation*}
		Die Koeffizientenmatrix  $A(t)$ sei stetig auf $I=(a,b)$. Dann existiert,f"ur die obige Gleichung, auf $I$ eine Fundamentalmatrix von $n$ L"osungen.
		
		\textbf{Beweisidee}: \\
		W"ahle $n$ beliebige, voneinander linear unabh"angige Anfangswerte, f"ur ein festes $t_0 \in I$. Dann gibt es
		eine eindeutige L"osung zu jedem der AWP auf ganz $I$. 
		Es gen"ugt zu zeigen, dass die L"osungen f"ur ein $t$ 
		linear unabh"angig sind. Das gilt aber f"ur $t_0$.
	\end{field}
		
	\begin{field}  % Question
		Wie lautet der Satz "uber die Bedeutung der Fundamentalmatrix?
	\end{field}
	
	\begin{field}  % Answer
		Es g"abe ein homogenes lineares DGL-System  ersten Ordnung:
		\begin{equation*}
		\dot{\vec{x}} = A(t)\vec{x} 
		\end{equation*}
		Ist $\Phi(t)$ eine Fundamentalmatrix von dem DGL-System auf $(a,b)$, dann ist die Allgemeine L"osung der Gleichung auf dem Gebiet $G=\qty{(t,x) \ | \ t\in(a,b), \norm{x} < \infty}$:
		\begin{equation*}
			x(t)= \Phi(t) \cdot c
		\end{equation*}
		wobei $c$ ein beliebiger Vektor ist. 
	\end{field}
\end{note}
	\begin{note}
		\xplain{UUID}
		\tags{LDGL, inhomogen,3.8.7}
		
		\begin{field}  % Question
			Wie findet man die L"osung eines inhomogenen linearen  DGL-System, falls die Fundamentalmatrix $\Phi$ des zugeh"origen homogenen DGL-System schon bekannt ist?
		\end{field}
		
		\begin{field}  % Answer
			Die L"osung hat die Form:
			\begin{equation*}
				x(t) = \underbrace{\Phi\cdot c}_{\text{allg. Lsg.  der hom. Glg.}} + \underbrace{\Psi}_{\text{eine spezielle Lsg. der inhom. Glg.}}
			\end{equation*}
			Sie kann mittels Variation der Konstanten bestimmt werden. Es gilt die folgende L"osungsformel:
			\begin{equation*}
				x(t) = \Phi(t)\qty[c + \int_{t_0}^{t}\Phi^{-1}(t')f(t')\dif t']
			\end{equation*}
		\end{field}
	\end{note}
	\begin{note}
		\xplain{UUID}
		\tags{LDGLS, konstante-koeffizienten, 3.9}
		
		\begin{field}  % Question
			Es g"abe einen linearen DGL-System der Dimension $n$
			\textbf{mit konstanten Koeffizienten}.
			\begin{equation*}
				\dot{x} = Ax
			\end{equation*}
			Wie l"ost man  das System mit der Ansatzmethode? 
		\end{field}
		
		\begin{field}  % Answer
	
			\textbf{Ansatz:} $x(t)= c \cdot e^{\lambda t}$,
				$\quad \lambda\in \C{}, c\in \C{n}$.
			
			
			$\Rightarrow$ Eigenwertgleichung:
			\begin{equation*}
				Ac = \lambda c \Rightarrow \text{det}(A - \lambda \mathbb{1}) \overset{!}{=} 0
			\end{equation*}
			Bestimme die Eigenwerte und Eigenvektoren.
			
			Dann sind $x_i = c_ie^{\lambda_i t}$ L"osungen.
		\end{field}
		\end{note}
		\begin{note}
			\xplain{UUID}
			\tags{LDGLS, konstante-koeffizienten, 3.9}
			
			\begin{field}  % Question
				Es g"abe einen linearen homogenen  DGL-System der Dimension $n$
				\textbf{mit konstanten Koeffizienten}
				$\dot{x} = Ax$.

				 Was gilt f"ur L"osungen, die mit der Ansatzmethode bestimmt wurden, falls $\A \in \C{n\times n}$ und was gilt f"ur  $\A \in \R{n\times n}$?
			\end{field}
			
			\begin{field}  % Answer
				$A,c_i, \lambda_i$,komplex: Man bekommt ein Fundamentalsystem falls $A$ $n$ verschiedene Eigenwerte hat.
				$A$ reellwertig:
				\begin{itemize}
					\item Fundamentalsystem wenn $A$ $n$ lin. unab. Eigenvektoren besitzt. 
					\item Ein komplexer Eigenwert tritt immer gemeinsam mit einem dazu c.c. auf.
					\item Aus einer komplexen L"osung $x(t)$ bekommt man zwei reelle L"osungen $u(t)= \real (x(t))$ und $v(t)=\imaginary (x(t))$.
				\end{itemize} 
			\end{field}
		\end{note}
		\begin{note}
			\xplain{UUID}
			\tags{LDGLS, konstante-koeffizienten, 3.9}
			
			\begin{field}  % Question
				Es g"abe einen homogenen linearen DGL-System der Dimension $n$
				\textbf{mit konstanten Koeffizienten}
				$\dot{x} = Ax$. Au"serdem g"abe es zu dem System ein AWP mit $x(t_0)=x_0$.
				Dann ist $e^At$...
			\end{field}
			
			\begin{field}  % Answer
				... Fundamentalmatrix von dem DGL-System auf \\ ${I=(-\infty, \infty)}$.
				
				Das Anfangswertproblem hat die L"osung:
				\begin{equation*}
					x(t) = e^{(t-t_0) A} \cdot x_0
				\end{equation*}  
			\end{field}
				
			\begin{field}  % Question
				Es g"abe ein AWP zu einem inhomogenen linearen DGL-System der Dimension $n$
				\textbf{mit konstanten Koeffizienten}
				$\dot{x} = Ax+ f(t), x(t_0)=x_0$.
				Wie berechnet man die L"osung?
			\end{field}
			
			\begin{field}  % Answer
				Es gilt die folgende L"osungsformel:
				\begin{equation}
					x(t) = e^{(t-t_0)A} x_0 + \int_{t_0}^{t}
					e^{(t-t')A}f(t')\dif t'
				\end{equation}
			\end{field}
		\end{note}
\end{document}