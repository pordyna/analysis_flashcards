% !TeX spellcheck = de_DE
% !TEX encoding = UTF-8 Unicode
%%%%%%%%%%%%%%%%%%%%%%%%%%%%%%%%%%%%%%%%%%%%%%%%%%%%
% The first part of the header needs to be copied
%       into the note options in Anki.
%%%%%%%%%%%%%%%%%%%%%%%%%%%%%%%%%%%%%%%%%%%%%%%%%%%%

% layout in Anki:
\documentclass[9pt]{article}
\usepackage[a4paper]{geometry}
\geometry{paperwidth=.5\paperwidth,paperheight=100in,left=2em,right=2em,bottom=1em,top=2em}
\pagestyle{empty}
\setlength{\parindent}{0in}
 
% hyphenation:
\usepackage[ngerman]{babel}

% encoding:
\usepackage[T1]{fontenc}
\usepackage[utf8]{inputenc}
\usepackage{lmodern}

% packages:
\usepackage{parskip}
\usepackage[free-standing-units=true]{siunitx}
% Writng si units, numbers, list of numbers etc.
\usepackage{gensymb}
% Unified typseting of units outside of siuitx
\usepackage{amsmath}
\usepackage{amsfonts}  
% Math features
\usepackage{esdiff}
% Typeseting of (partial)derivatives 
\usepackage{commath}
% More  derivatives. Not so nice like esdiif but adds
% \dif comand for upright d in math mode.
\usepackage{bm}
% Bold font in math mode
\usepackage{esint}
% Some fancy integrals signs. Mutilpe integrals
\usepackage{enumerate}
% Different styles for enumerate lists
\usepackage{multirow}
% More advanced tabular
\usepackage{physics}
% Lots of usefull comannds for physicists. Vektors, nablas etc.
\usepackage{amssymb, xfrac, bbold}
% extra fonts and symbols 
\usepackage{mathtools}
% extension to amsmath, fixes, meany new tool
\usepackage{isotope}
\usepackage{empheq}
%commands:
\usepackage{ifthen}

\newcommand*{\tran}{^{\mkern-1.5mu\mathsf{T}}}
\newcommand{\Rn}{\mathbb{R}^n}
\newcommand{\Rk}{\mathbb{R}^k}
\newcommand{\R}[1]{%
	\ifthenelse{\equal{#1}{}}
	{\mathbb{R}}
	{\mathbb{R}^{#1}}}%
\newcommand{\C}[1]{%
	\ifthenelse{\equal{#1}{}}
	{\mathbb{C}}
	{\mathbb{C}^{#1}}}%	
\renewcommand{\vec}[1]{\underline{#1}}
\DeclarePairedDelimiter{\innerprod}\langle\rangle
\newcommand{\Fr}{\mathcal{F}}
\newcommand{\Hi}{\mathcal{H}}

%%%%%%%%%%%%%%%%%%%%%%%%%%%%%%%%%%%%%%%%%%%%%%%%%%%%
% Following part of header NOT to be copied into
%            the note options in Anki.
%          ! Anki will throw an errow !
%%%%%%%%%%%%%%%%%%%%%%%%%%%%%%%%%%%%%%%%%%%%%%%%%%%%%
%
%  pdf layout:
%
\geometry{paperheight=74.25mm}
\usepackage{pgfpages}
\pagestyle{empty}
\pgfpagesuselayout{8 on 1}[a4paper,border shrink=0cm]
\makeatletter
\@tempcnta=1\relax
\loop\ifnum\@tempcnta<9\relax
\pgf@pset{\the\@tempcnta}{bordercode}{\pgfusepath{stroke}}
\advance\@tempcnta by 1\relax
\repeat
\makeatother
% 
%  notes, fields, tags:
%
\newcommand{\xfield}[1]{
        #1\par
        \vfill
        {\tiny\texttt{\parbox[t]{\textwidth}{\localtag\\\globaltag\hfill\uuid}}}
        \newpage}
\newenvironment{field}{}{\newpage}
\newif\ifnote
\newenvironment{note}{\notetrue}{\notefalse}
\newcommand{\localtag}{}
\newcommand{\globaltag}{}
\newcommand{\uuid}{}
\newcommand{\tags}[1]{
    \ifnote 
        \renewcommand{\localtag}{#1}
    \else
        \renewcommand{\globaltag}{#1}
    \fi 
    }
\newcommand{\xplain}[1]{\renewcommand{\uuid}{#1}}
%
%%%%%%%%%%%%%%%%%%%%%%%%%%%%%%%%%%%%%%%%%%%%%%%%%%%%
% The following line again needs to be copied 
% into Anki:
\begin{document}
%%%%%%%%%%%%%%%%%%%%%%%%%%%%%%%%%%%%%%%%%%%%%%%%%%%%

\tags{mathe2::2sem::hilbertraum, beschrankte-lineare-operatoren}

\begin{note}
	\xplain{UUID}
	\tags{}
	
	\begin{field}  % Question
		Wann ist eine linearer Operator aus $\Hi$ \textit{dicht definiert}?
	\end{field}
	
	\begin{field}  % Answer
		Wenn $D(A)$ \textit{Definitionsbereich} dicht in $\Hi$ ist.
	\end{field}
\end{note}
	\begin{note}
		\xplain{UUID}
		\tags{6.1.3}
		
		\begin{field}  % Question
			Wie ist ein beschr"ankter Operator definiert?	
		\end{field}
		
		\begin{field}  % Answer
			Sei $A: D(A) \rightarrow \Hi$ ein linearer Operator. $A$ hei"st beschr"ankt, falls
			f"ur jede beschr"ankte Menge $M\subset D(A)$ gilt, dass $A[M]$ beschr"ankt ist.
		\end{field}
		
			
		\begin{field}  % Question
			Sei $A: D(A) \rightarrow \Hi$ ein linearer Operator. $A$ hei"st stetig, falls
		\end{field}
		
		\begin{field}  % Answer
			f"ur jede Folge $(\varphi_n)$ in $D(A)$ mit $\varphi_n \rightarrow \varphi \in D(A)$ gilt, dass
			\begin{equation*}
				A\varphi_n \rightarrow A\varphi
			\end{equation*}
		\end{field}
	\end{note}
	\begin{note}
		\xplain{UUID}
		\tags{6.1.4, satz}
		
		\begin{field}  % Question
			Sei $A:D(A) \rightarrow \Hi$ ein linearer Operator. Welche Aussagen sind "aquivalent zu
			\begin{itemize}
				\item $A$ ist beschr"ankt ?
			\end{itemize}
		\end{field}
		
		\begin{field}  % Answer
			\begin{itemize}
				\item $\exists c>0 : \forall \varphi \in D(A) \norm{A\phi} \leq c\norm{phi}$,
				\item $A$ ist stetig,
				\item $A$ ist stetig in 0.
			\end{itemize}
		\end{field}
		
		\begin{field}  % Question
			Wann l"asst sich ein linearer Operator $A$ auf ganz $\Hi$ fortsetzen?
		\end{field}
		
		\begin{field}  % Answer
			Falls $D(A)$ dicht in $\Hi$ und $A$ stetig, l"asst sich $A$ \textbf{stetig} auf $\Hi$ fortsetzen.
		\end{field}
	\end{note}
	\begin{note}
		\xplain{UUID}
		\tags{6.1.5, 6.1.6, 6.1.7, operatornorm}
		
		\begin{field}  % Question
			Was ist $B(\Hi)$? 
		\end{field}
		
		\begin{field}  % Answer
			$B(\Hi)$ ist der Vektorraum aller \textbf{beschr"ankten} linearen Operatoren auf $\Hi$.
			Es ist ein Vektorraum ,da f"ur ein skalar $c$ und zwei beschr"ankte Operatoren $A_1, A_2$
			$A_1 + A_2$ und $cA$ auch beschr"ankt sind. 
		\end{field}
		
			
		\begin{field}  % Question
			Wie ist die Operatornorm auf $B(\Hi)$ definiert? 
		\end{field}
		\begin{field}  % Answer
			\begin{equation*}
				\norm{T} \coloneqq \ \text{inf}\qty{c>0 \ | \ \forall \varphi \in \Hi: \norm{T\varphi} \leq c\norm{\varphi}}
			\end{equation*}
		\end{field}
			
		\begin{field}  % Question
			Was besagt der Satz Norm auf $B(\Hi)$, au"ser dass die definierte Operatornorm tats"achlich einer Norm auf $(\Hi)$ ist? 
		\end{field}
		
		\begin{field}  % Answer
			$\qty(B(\Hi), \norm{\cdot})$ ist ein Banach-Raum.
			
			Es gilt die \textit{Submultiplikativit"at} der Norm, also $\norm{AB} \leq \norm{A}\norm{B}$.
		\end{field}
			
		\begin{field}  % Question
			Welche "aquivalente zu 
			\begin{equation*}
			\norm{T} \coloneqq \ \text{inf}\qty{c <0 \ | \ \forall \varphi \in \Hi
			\norm{T\varphi} \leq \norm{\varphi}}
			\end{equation*}
		 Charakterisierungen der Operatornorm auf $B(\Hi)$ gibt es (Es sind 4.)?
		\end{field}
		
		\begin{field}  % Answer
			\begin{align*}
				\norm{T} =& \ \text{sup}\qty{\norm{T\varphi} \ | \ \varphi \in \Hi, \norm{\varphi} \leq 1} \\
				\norm{T} =& \ \text{sup}\qty{\norm{T\varphi} \ | \ \varphi \in \Hi, \norm{\varphi} = 1} \\
				\norm{T} =& \ \text{sup}\qty{\norm{T\varphi} \ | \ \varphi \in \Hi, \norm{\varphi} < 1} \\
				\norm{T} =& \ \text{sup}\qty{\abs{\innerprod{\psi, T\varphi}} \ | \ \psi, \varphi \in \Hi, \qty(\norm{\psi} \leq 1 \wedge \norm{\varphi} \leq 1)} \\
			\end{align*}
		\end{field}
	\end{note}
	\begin{note}
		\xplain{UUID}
		\tags{6.1.8}
		
		\begin{field}  % Question
			Ein Operator aus $B(\Hi)$ hei"st \textit{endlich-dimensional}, wenn ...
		\end{field}
		
		\begin{field}  % Answer
			... der Bildbereich von diesem Operator endlich-dimensional ist.
		\end{field}
	\end{note}
	\begin{note}
		\xplain{UUID}
		\tags{6.1.9,6.1.10, inverse}
		
		\begin{field}  % Question
			Wann existiert der inverse Operator zu einem linearen Operator $A$?  
		\end{field}
		
		\begin{field}  % Answer
			Falls $A: D(A) \rightarrow \text{im}A$ injektiv. Dann ist $A^{-1}$ auch linear.
		\end{field}
			
		\begin{field}  % Question
			Was besagt der Satz "uber den inversen Operator?
		\end{field}
		
		\begin{field}  % Answer
			Wenn $A\in B(\Hi), \ \text{im}A=\Hi$ und $A^{-1}$ existiert, dann ist $A^{-1} \in B(\Hi)$.
		\end{field}
	\end{note}
	\begin{note}
		\xplain{UUID}
		\tags{6.2.1, adjungierte-operator}
		
		\begin{field}  % Question
			Was besagt der Sagt "uber den adjungierten Operator?
		\end{field}
		
		\begin{field}  % Answer
			Zu jedem $T \in B(\Hi)$ existiert genau ein $T^*\in B(\Hi)$ mit folgender Eigenschaft:
			\begin{equation*}
				\forall \varphi, \psi \in \Hi: \quad \innerprod{\varphi, T\psi} = \innerprod{T^*\varphi, \psi}
			\end{equation*}
			$T^*$ hei"st der zu $T$ adjungierte Operator. Es gilt: $\norm{T}= \norm{T^*}$
		\end{field}
	\end{note}
	\begin{note}
		\xplain{UUID}
		\tags{6.2.2}
		
		\begin{field}  % Question
			$(B(\Hi), \norm{\cdot})$ ist eine $C^*$-Algebra mit Eins-Element, d.h. ...
		\end{field}
		
		\begin{field}  % Answer
			...
			\footnotesize
			\begin{enumerate}[i)]
				\item $B(\Hi)$ ist ein $\C{}$-Vektorraum mit einer \textbf{Multiplikation} (assoziative, bilineare Abbildung)
				\begin{equation*}
					B(\Hi) \times B(\Hi) \ni (S,T) \mapsto S \circ T
				\end{equation*}
				Als Vektorraum mit \textbf{assoziativer bilinearer} Abbildung 
				ist $B(\Hi)$ eine Abbildung. Das Eins-Element dieser Algebra ist Einheitsoperator $\mathbb{1}$.
				\item $(B(\Hi), \norm{\cdot})$ ist eine \textbf{normierte} Algebra, d.h. $\norm{\cdot}$ ist \textbf{submultiplikativ}:
				\begin{equation*}
					\forall S,T \in B(\Hi): \norm{ST} \leq \norm{S} \norm{T}
				\end{equation*}
				Es ist eine \textbf{Banach-Algebra} (also vollst"andig und normiert).
				\item $B(\Hi)$ ist eine \textbf{$^*$-Algebra}.
				\item Die Norm erf"ullt die \textbf{${C^*}$-Algebra}, d.h. $\forall T \in B(\Hi): \norm{T^*T} = \norm{T}^2$.
			\end{enumerate}
		\end{field}
			
		\begin{field}  % Question
			$B(\Hi)$ ist eine \textbf{$^*$-Algebra}, d.h. ...
		\end{field}
		
		\begin{field}  % Answer
			... es gibt eine Abbildung $*: B(\Hi) \rightarrow B(\Hi), \ \ T\mapsto T^*$ mit
			
			$\forall S,T \in B(\Hi), \forall \lambda \in \C{}:$
			\begin{enumerate}
				\item $\qty(\lambda S + T)^* = \bar{\lambda}S^* + T^*$,
				\item $\qty(ST)^* = T^*S^*$,
				\item $S^{**} = S$
			\end{enumerate}
		Eine solche Abbildung hei"st \textit{Involution}.
		\end{field}
	\end{note}
\end{document}
