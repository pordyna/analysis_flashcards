% !TEX encoding = UTF-8 Unicode
%%%%%%%%%%%%%%%%%%%%%%%%%%%%%%%%%%%%%%%%%%%%%%%%%%%%
% The first part of the header needs to be copied
%       into the note options in Anki.
%%%%%%%%%%%%%%%%%%%%%%%%%%%%%%%%%%%%%%%%%%%%%%%%%%%%

% layout in Anki:
\documentclass[11pt]{article}
\usepackage[a4paper]{geometry}
\geometry{paperwidth=.5\paperwidth,paperheight=100in,left=2em,right=2em,bottom=1em,top=2em}
\pagestyle{empty}
\setlength{\parindent}{0in}
 
% hyphenation:
\usepackage[ngerman]{babel}

% encoding:
\usepackage[T1]{fontenc}
\usepackage[utf8]{inputenc}
\usepackage{lmodern}

% packages:
\usepackage[free-standing-units=true]{siunitx}
% Writng si units, numbers, list of numbers etc.
\usepackage{gensymb}
% Unified typseting of units outside of siuitx
\usepackage{amsmath}
\usepackage{amsfonts}  
% Math features
\usepackage{esdiff}
% Typeseting of (partial)derivatives 
\usepackage{commath}
% More  derivatives. Not so nice like esdiif but adds
% \dif comand for upright d in math mode.
\usepackage{bm}
% Bold font in math mode
\usepackage{esint}
% Some fancy integrals signs. Mutilpe integrals
\usepackage{enumerate}
% Different styles for enumerate lists
\usepackage{multirow}
% More advanced tabular
\usepackage{physics}
% Lots of usefull comannds for physicists. Vektors, nablas etc.
\usepackage{amssymb, xfrac}
% extra fonts and symbols 
\usepackage{mathtools}
% extension to amsmath, fixes, meany new tool
\usepackage{isotope}

%%%%%%%%%%%%%%%%%%%%%%%%%%%%%%%%%%%%%%%%%%%%%%%%%%%%
% Following part of header NOT to be copied into
%            the note options in Anki.
%          ! Anki will throw an errow !
%%%%%%%%%%%%%%%%%%%%%%%%%%%%%%%%%%%%%%%%%%%%%%%%%%%%%
%
%  pdf layout:
%
\geometry{paperheight=74.25mm}
\usepackage{pgfpages}
\pagestyle{empty}
\pgfpagesuselayout{8 on 1}[a4paper,border shrink=0cm]
\makeatletter
\@tempcnta=1\relax
\loop\ifnum\@tempcnta<9\relax
\pgf@pset{\the\@tempcnta}{bordercode}{\pgfusepath{stroke}}
\advance\@tempcnta by 1\relax
\repeat
\makeatother
% 
%  notes, fields, tags:
%
\newcommand{\xfield}[1]{
        #1\par
        \vfill
        {\tiny\texttt{\parbox[t]{\textwidth}{\localtag\\\globaltag\hfill\uuid}}}
        \newpage}
\newenvironment{field}{}{\newpage}
\newif\ifnote
\newenvironment{note}{\notetrue}{\notefalse}
\newcommand{\localtag}{}
\newcommand{\globaltag}{}
\newcommand{\uuid}{}
\newcommand{\tags}[1]{
    \ifnote 
        \renewcommand{\localtag}{#1}
    \else
        \renewcommand{\globaltag}{#1}
    \fi 
    }
\newcommand{\xplain}[1]{\renewcommand{\uuid}{#1}}
%
%%%%%%%%%%%%%%%%%%%%%%%%%%%%%%%%%%%%%%%%%%%%%%%%%%%%
% The following line again needs to be copied 
% into Anki:
\begin{document}
%%%%%%%%%%%%%%%%%%%%%%%%%%%%%%%%%%%%%%%%%%%%%%%%%%%%

\tags{mathe2::1sem::untermanigfaltigkeiten}

\begin{note}
  \xplain{2cf30783-9699-47b2-b8a6-ecc059beea33}
  \tags{definition}
  \xfield{Wann nennt man eine Abbildung \textit{Diffeomorphismus}?}
  \begin{field}
    Seien $U,V \subseteq \mathbb{R}^n$ offen. Eine Abbildung $\Psi : U \rightarrow V$ hei"st \textit{Diffeomorphismus},
    falls $\Psi$ bijektiv und sowohl $\Psi$ als auch $\Psi^{-1} : V\rightarrow U$ stetig diff'bar sind.
  \end{field}
\end{note}
\begin{note}
	\xplain{UUID}
	\tags{definition, 1.1.1}
	\xfield{Was hei"st es, dass  eine Abbildung \textit{regul"ar} ist?}
	\begin{field}
		Sei $T\subseteq \mathbb{R}^n$ offen. Eine Abbildung $\Phi : T \rightarrow \mathbb{R}^n$ hei"st \textit{regul"ar}, falls $\Phi$ injektiv und stetig diff'bar ist, $\Phi'$ den Rang $k$ hat und $\Phi^{-1} : \Phi\qty[T] \rightarrow T$ stetig ist.
	\end{field}
	
\end{note}

\begin{note}
	\xplain{UUID}
	\tags{definition, 1.1.2}
	\begin{field}Eine Teilmenge $M \subseteq \mathbb{R}^n$ hei"st \textit{k-dimensionale Untermannigfaltigkeit} (UM) des
		$\mathbb{R}^n$, wenn...
	\end{field}
	\begin{field}
		$\forall_a \in M$ $\exists$ offene Mengen $U, V \subseteq \mathbb{R}^n$ mit $a\in U$ 
		(d.h. $U$ ist offene Umgebung von $a$) und ein Diffeomorphismus $\Psi : U \rightarrow V$ so, dass
		\begin{align*}
		\Psi\left[U \cap M \right] &= \left\lbrace (y_1, \dots, y_n) \in V; y_{k+1} = \dots = y_n = 0\right\rbrace \\
		&= V\cap \qty(\mathbb{R}^k \times 0_{n-k})
		\end{align*}
	\end{field}
	\xfield{Jede $\qty(n-1)$-dim. Untermannigfaltigkeit des $\mathbb{R}^n$ hei"st...}
	\xfield{\textit{Hyperfl"ache}.}
\end{note}

\end{document}
