% !TEX encoding = UTF-8 Unicode
%%%%%%%%%%%%%%%%%%%%%%%%%%%%%%%%%%%%%%%%%%%%%%%%%%%%
% The first part of the header needs to be copied
%       into the note options in Anki.
%%%%%%%%%%%%%%%%%%%%%%%%%%%%%%%%%%%%%%%%%%%%%%%%%%%%

% layout in Anki:
\documentclass[9pt]{article}
\usepackage[a4paper]{geometry}
\geometry{paperwidth=.5\paperwidth,paperheight=100in,left=2em,right=2em,bottom=1em,top=2em}
\pagestyle{empty}
\setlength{\parindent}{0in}
 
% hyphenation:
\usepackage[ngerman]{babel}

% encoding:
\usepackage[T1]{fontenc}
\usepackage[utf8]{inputenc}
\usepackage{lmodern}

% packages:
\usepackage{parskip}
\usepackage[free-standing-units=true]{siunitx}
% Writng si units, numbers, list of numbers etc.
\usepackage{gensymb}
% Unified typseting of units outside of siuitx
\usepackage{amsmath}
\usepackage{amsfonts}  
% Math features
\usepackage{esdiff}
% Typeseting of (partial)derivatives 
\usepackage{commath}
% More  derivatives. Not so nice like esdiif but adds
% \dif comand for upright d in math mode.
\usepackage{bm}
% Bold font in math mode
\usepackage{esint}
% Some fancy integrals signs. Mutilpe integrals
\usepackage{enumerate}
% Different styles for enumerate lists
\usepackage{multirow}
% More advanced tabular
\usepackage{physics}
% Lots of usefull comannds for physicists. Vektors, nablas etc.
\usepackage{amssymb, xfrac}
% extra fonts and symbols 
\usepackage{mathtools}
% extension to amsmath, fixes, meany new tool
\usepackage{isotope}

%commands:
\newcommand*{\tran}{^{\mkern-1.5mu\mathsf{T}}}
\newcommand{\Rn}{\mathbb{R}^n}
\newcommand{\Rk}{\mathbb{R}^k}
\newcommand{\R}[1]{\mathbb{R}^#1}
\newcommand{\C}[1]{\mathbb{Cs}^#1} 
%%%%%%%%%%%%%%%%%%%%%%%%%%%%%%%%%%%%%%%%%%%%%%%%%%%%
% Following part of header NOT to be copied into
%            the note options in Anki.
%          ! Anki will throw an errow !
%%%%%%%%%%%%%%%%%%%%%%%%%%%%%%%%%%%%%%%%%%%%%%%%%%%%%
%
%  pdf layout:
%
\geometry{paperheight=74.25mm}
\usepackage{pgfpages}
\pagestyle{empty}
\pgfpagesuselayout{8 on 1}[a4paper,border shrink=0cm]
\makeatletter
\@tempcnta=1\relax
\loop\ifnum\@tempcnta<9\relax
\pgf@pset{\the\@tempcnta}{bordercode}{\pgfusepath{stroke}}
\advance\@tempcnta by 1\relax
\repeat
\makeatother
% 
%  notes, fields, tags:
%
\newcommand{\xfield}[1]{
        #1\par
        \vfill
        {\tiny\texttt{\parbox[t]{\textwidth}{\localtag\\\globaltag\hfill\uuid}}}
        \newpage}
\newenvironment{field}{}{\newpage}
\newif\ifnote
\newenvironment{note}{\notetrue}{\notefalse}
\newcommand{\localtag}{}
\newcommand{\globaltag}{}
\newcommand{\uuid}{}
\newcommand{\tags}[1]{
    \ifnote 
        \renewcommand{\localtag}{#1}
    \else
        \renewcommand{\globaltag}{#1}
    \fi 
    }
\newcommand{\xplain}[1]{\renewcommand{\uuid}{#1}}
%
%%%%%%%%%%%%%%%%%%%%%%%%%%%%%%%%%%%%%%%%%%%%%%%%%%%%
% The following line again needs to be copied 
% into Anki:
\begin{document}
%%%%%%%%%%%%%%%%%%%%%%%%%%%%%%%%%%%%%%%%%%%%%%%%%%%%

\tags{mathe2::1sem::untermanigfaltigkeiten}

\begin{note}
  \xplain{2cf30783-9699-47b2-b8a6-ecc059beea33}
  \tags{definition}
  \xfield{Wann nennt man eine Abbildung \textit{Diffeomorphismus}?}
  \begin{field}
    Seien $U,V \subseteq \mathbb{R}^n$ offen. Eine Abbildung $\Psi : U \rightarrow V$ hei"st \textit{Diffeomorphismus},
    falls $\Psi$ bijektiv und sowohl $\Psi$ als auch $\Psi^{-1} : V\rightarrow U$ stetig diff'bar sind.
  \end{field}
\end{note}
\begin{note}
	\xplain{212ab856-acf1-4148-9516-48d4263c7ccf}
	\tags{definition, 1.1.1}
	\xfield{Was hei"st es, dass  eine Abbildung \textit{regul"ar} ist?}
	\begin{field}
		Sei $T\subseteq \mathbb{R}^k$ offen. Eine Abbildung $\Phi : T \rightarrow \mathbb{R}^n$ hei"st \textit{regul"ar}, falls $\Phi$ injektiv und stetig diff'bar ist, $\Phi'$ den Rang $k$ hat und $\Phi^{-1} : \Phi\qty[T] \rightarrow T$ stetig ist.
	\end{field}
	
\end{note}

\begin{note}
	\xplain{e673a838-a7d8-428f-8d73-159052a44c8d}
	\tags{definition, 1.1.2}
	\begin{field}Eine Teilmenge $M \subseteq \mathbb{R}^n$ hei"st \textit{k-dimensionale Untermannigfaltigkeit} (UM) des
		$\mathbb{R}^n$, wenn...
	\end{field}
	\begin{field}
		$\forall_a \in M$ $\exists$ offene Mengen $U, V \subseteq \mathbb{R}^n$ mit $a\in U$ 
		(d.h. $U$ ist offene Umgebung von $a$) und ein Diffeomorphismus $\Psi : U \rightarrow V$ so, dass
		\begin{align*}
		\Psi\left[U \cap M \right] &= \left\lbrace (y_1, \dots, y_n) \in V; y_{k+1} = \dots = y_n = 0\right\rbrace \\
		&= V\cap \qty(\mathbb{R}^k \times 0_{n-k})
		\end{align*}
	\end{field}
	\xfield{Jede $\qty(n-1)$-dim. Untermannigfaltigkeit des $\mathbb{R}^n$ hei"st...}
	\xfield{\textit{Hyperfl"ache}.}	
	\xfield{Was ist eine \textit{Hyperfl"ache}?}
	\xfield{Jede $\qty(n-1)$-dim. Untermannigfaltigkeit des $\mathbb{R}^n$.}
\end{note}
\begin{note}
	\tags{satz, 1.1.3}
	\xplain{9e952d37-701a-44bc-9c3d-4bf6ed35b42b}
	\begin{field}
		$M$ ist eine $k$-dim. Untermannigfaltigkeit des $\mathbb{R}^n$. Wie l"asst sich $M$ als \textit{Nullstellenmenge} 	definieren? 
	\end{field}
	\begin{field}
		$\forall_a \in M \ \exists$ eine offene Umgebung $U \subseteq \mathbb{R}^{n}$ von $a$ und $n-k$ stetig diff'bare Funktionen $f_1, \dots , f_{n-k} : U \rightarrow \mathbb{R}$ so, dass
		\begin{equation*}
			M \cap U = \qty{x \in U; f_1(x) = \dots = f_{n-k}(x) = 0}
		\end{equation*}
		und
		\begin{equation*}
			\text{Rang}\diffp[]{\qty(f_1,\dots, f_{n-k})}{\qty(x_1,\dots,x_n)} = n-k
		\end{equation*}
		Das hei"st, dass die $f_i$ linear unabh"angig sind.
	\end{field}
	\begin{field}
		$M$ ist eine $k$-dim. Untermannigfaltigkeit des $\mathbb{R}^n$.
		Wie l"asst sich $M$ als \textit{Graph} definieren?
	\end{field}
	\begin{field}
		$\forall_a \in M$ gibt es (evt. nach geeigneter Umnummerierung der Koordinaten) offene Umgebungen $U' \in \mathbb{R}^k$ von $a'=\qty(a_1,\dots ,a_k)$, $U''\subseteq \mathbb{R}^{n-k}$ von $a'' = \qty(a_{k+1}, \dots , a_n)$, sowie eine stetig diff'bare Abbildung $g: U' \rightarrow U ''$ so, dass 
		\begin{equation*}
			M \cap \qty(U' \times U'') = \qty{(x', x'') \in U' \times U''; x'' = g(x')} = G(g)
		\end{equation*}
	\end{field}
	\begin{field}
		$M$ ist eine $k$-dim. Untermannigfaltigkeit des $\mathbb{R}^n$. Wie l"asst sich $M$ mittels der
		\textit{Parameterdarstellung} definieren?
	\end{field}
	\begin{field}
		$\forall_a \in M \ \exists$ eine offene Umgebung $U\subseteq \mathbb{R}^n$ von $a$, eine offene Menge $T \subseteq \mathbb{R}^k$, sowie eine regul"are Abbildung $\Phi:T\rightarrow \mathbb{R}^n \qq{mit} \Phi(T) = U \cap M = \colon \W $
	\end{field}
\end{note}

\begin{note}
	\xplain{5b549aaf-88d1-4853-b2f0-cd81b664ca36}
	\tags{definition, 1.1.4}
	\xfield{Wann spricht man von einer \textit{globalen Parametrisierung} einer UM $M$?}
	\begin{field}
		$\Phi$ und $T$ wie in der Definition der Parameterdarstellung. $(\Phi, T)$ hei"st \textit{globale Parametrisierung} falls
		\begin{equation*}
			\Phi(T) = M
		\end{equation*}
		Sonst spricht man von einer lokalen Parametrisierung.
	\end{field}
	\begin{field}
		$\Psi = \Phi^{-1}$, $W=T\cap M$. $\Phi$, $T$ und $a$ wie in der Definition der Parameterdarstellung. Wie nennt man $(\Psi,W)$ und wie hei"sen die Komponenten von $(t_1, \dots, t_k)\coloneqq \Psi(a)$?
	\end{field}
	\begin{field}
		$(\Psi,W)$ hei"st \textit{Karte um} $a$ und die Komponenten  des Vektors $\Psi(a)$ hei"sen \textit{lokale Koordinaten von} $a$. 
	\end{field}
	\xfield{$M$ eine UM. Ein \textit{Atlas von }$M$ ist..}
	\begin{field}
		ein System von Karten, das $M$ "uberdeckt.
	\end{field}
\end{note}
	\begin{note}
		\xplain{c15dec62-ecbb-4a35-810c-b5b55fe4e76f}
		\tags{satz, kartenwechsel, 1.1.5}
		\xfield{Wie lautet der Satz "uber die Parametertransformation (Kartenwechsel)? }
		\begin{field}
			Seien $M$ eine $k$-dimensionale UM des $\mathbb{R}^n$, $a \in M$,$(\Psi_1, W_1)$, $(\Psi_2, W_2)$ zwei Karten (($\Phi_1, T_1$), ($\Phi_2,T_2$) sind die entsprechende Parametrisierungen.) um $a \in M$ mit $W \coloneqq W_1 \cap W_2 \neq \emptyset$.
			
			
			Dann sind $S_i = \Psi_i(W)$ offene Teilmengen von $T$; und $h\coloneqq \Psi_2 \circ \Psi_1^{-1} : S_1 \rightarrow S_2$ ist ein Diffeomorphismus.
			
			
			Die Abbildung $h$ hei"st \textit{Kartenwechsel}.
		\end{field}
	\end{note}
%%%% 1.2 Tangentialr"aume an UM. %%%%
\begin{note}
	\xplain{f73c2014-1545-4520-a56b-c027b06b8bbe}
	\tags{definition, tangentialraum, 1.2.1}
	\begin{field}
		Sei $M$ eine UM des $\mathbb{R}^n$ und $a \in M $. Ein Vektor $v\in \mathbb{R}^n$ hei"st \textit{Tangentialvektor an $M$ in $a$}, wenn ...
	\end{field}
	\begin{field}
		es eine stetige, diff'bare Abbildung $\alpha: (-\varepsilon, \varepsilon) \rightarrow M$ gibt mit $\alpha(0) = a, \alpha'(0) = v$.
	\end{field}
	\xfield{Sei $M$ eine UM des $\mathbb{R}^n$ und $a \in M $. Wie ist ein \textit{Tangentialraum an $M$ in $a$} definiert?}
	\begin{field}
		Ein \textit{Tangentialraum an $M$ in $a$} ist die Menge aller Tangentialvektoren an $M$ in $a$ und wird mit $T_a(M)$ bezeichnet.
	\end{field}
	\begin{field}
		Sei $M$ eine UM des $\mathbb{R}^n$ und $a \in M $. Ein Vektor $w\in \mathbb{R}^n$ hei"st \textit{Normalenvektor an $M$ in $a$}, wenn ...
	\end{field}
	\begin{field}
		$\forall v\in T_a(M): w \perp v$ (d.h. orthogonal bzgl. des kanonischen Skalarprodukts im $\mathbb{R}^n$).
	\end{field}
	\xfield{Sei $M$ eine UM des $\mathbb{R}^n$ und $a \in M $. Wie ist ein \textit{Normalraum an $M$ in $a$} definiert?}
	\begin{field}
		Ein \textit{Normalraum an $M$ in $a$} ist die Menge aller Normalenvektoren an $M$ in $a$ und wird mit $N_a(M)$ bezeichnet.
	\end{field}
\end{note}
\begin{note}
		\xplain{f54d3b26-2bb6-4cb0-a2ec-986f52924e47}
		\tags{satz, tangentialraum, 1.2.2}
		\begin{field}
			Seien $M$ eine $k$-dimensionale UM des $\mathbb{R}^n$, $a\in M$. Wie l"asst sich eine Basis von $T_a(M)$ mittels einer Parameterdarstellung finden? 
		\end{field}
		\begin{field}
			$T_a(M)$ ist ein $k$-dimensionaler Vektorraum. Ist eine lokale Parameterdarstellung $(\Psi, T)$, also:
			\begin{equation*}
				T\subseteq \mathbb{R}^k, \Phi : T \rightarrow M \qq{und} c\in T \qq{mit} \Phi(c) = a,
			\end{equation*}	
			dann bilden die Vektoren
			\begin{equation*}
				\diffp[]{\Phi}{t_1}(c), \dots, \diffp[]{\Phi}{t_k}(c)
			\end{equation*}
			eine Basis von $T_a(M)$.
		\end{field}
		\begin{field}
			Wird $M$ lokal als Nullstellenmenge gegeben, wie l"asst sich eine Basis f"ur $N_a(M)$ finden?
		\end{field}
		\begin{field}
				$T_a(M)$ ist ein $k$-dimensionaler Vektorraum. Ist $M$ lokal als Nullstellenmenge gegeben (Beschreibung durch Gleichungen) mit $U\subseteq \mathbb{R}^n$,
			\begin{gather*}
				f = \qty(f_1,\dots f_{n-k}) : U \rightarrow \mathbb{R}^{n-k}, \\
				a \in M \cap U = \qty{x \in U; f(x) = 0}, \\
				\text{Rang}\diffp[]{(f_1,\dots,f_{n-k})}{(x_1,\dots x_k)}(a) = n-k. \\
			\end{gather*}
			Dann bilden die Vektoren $\text{grad}f_1(a),\dots,\text{grad}f_{n-k}(a)$ eine Basis f"ur $N_a(M)$.
		\end{field}
\end{note}

%%%%% 1.3 Integration skalarer Funtionen "uber UM. %%%%%

\begin{note}
	\xplain{b2af8cc4-0ef4-46d4-9247-8bf66f5cb955}
	\tags{definition, integration, 1.3.1}
	\begin{field}
		Sei $M$ eine 2-dimensionale UM des $\mathbb{R}^n$. Wie ist \textit{Inhalt von $M$} definiert?
	\end{field}
	\begin{field}
		Es g"abe eine Parameterdarstellung $\Phi$ : $T \rightarrow \Phi\qty[T] = M$, wobei $T$ offen und jordanmessbar, und die partiellen Ableitungen von $\Phi$ seien beschr"ankt auf $T$.
		
		Unter dem \textit{Inhalt von M} versteht man:
		\begin{equation*}
			\abs{M}\coloneqq \int_T\underbrace{\norm{\Phi_{t_1}(t_1,t_2) \cross \Phi_{t_2}(t_1,t_2)} \dif t_1 \dif t_2}_{\dif S}
		\end{equation*} 
		Man nennt $\dif S$ das \textit{(2-dim.) Fl"achenelement (bzgl. $\Phi$)}.
	\end{field}
	\xfield{Sei $M$ eine 2-dimensionale UM des $\mathbb{R}^n$. Wie definiert man $\int_M f\dif S$?}
	\begin{field}
		Es g"abe eine Parameterdarstellung $\Phi$ : $T \rightarrow \Phi\qty[T] = M$, wobei $T$ offen und jordanmessbar, und die partiellen Ableitungen von $\Phi$ seien beschr"ankt auf $T$. Sei $f : M \rightarrow \mathbb{R}$ eine beschr"ankte stetige Funktion.
		\begin{align*}
			\int_M f \dif S \coloneqq& \int_M f(x) \dif S(x) \\ \coloneqq& \int_T f\qty(\Phi\qty(t)) \cdot \norm{\Phi_{t_1}(t_1,t_2) \cross \Phi_{t_2}(t_1,t_2)}\dif t_1 \dif t_2
		\end{align*}
	\end{field}
\end{note}
\begin{note}
	\xplain{a1d2cd27-52ad-4d38-b96b-1ed983795210}
	\tags{definition, integration, 1.3.3}
	\xfield{Das Volumen des $k$-Parallelepipeds $P(a^{(1)}, \dots a^{(k)})$ ist definiert als ... }
	\begin{field}
		\begin{equation*}
			V_k(a^{(1)}, \dots a^{(k)})\coloneqq \sqrt{\text{det}(A\tran A)}
		\end{equation*}
	\end{field}
\end{note}
\begin{note}
	\xplain{9d5f259f-3436-4f42-9ee1-fd5f0277c1a5}
	\tags{definition, integration, 1.3.4}
	\begin{field}
		Seien$ (\Phi, T)$ eine lokale Parametrisierung, $t\in T$.
		Wie sind der metrischer Tensor von $\Phi$ ($g_{ij}$) und $g\coloneqq \text{det}(g_{ij})$ definiert?	
	\end{field}
	 \begin{field}
	 	\begin{equation*}
	 		(g_{ij}) \coloneqq (\Phi')\tran \Phi' = \qty(\left<\diffp[]{\Phi}{t_i}, \diffp[]{\Phi}{t_j}\right>)
	 	\end{equation*}
	 	\begin{equation*}
	 		g \coloneqq \text{det}(g_{ij}) = \text{det}\qty((\Phi')\tran \Phi')
	 	\end{equation*}
	 \end{field}
\end{note}
\begin{note}
	\xplain{20f0663e-1658-48dc-81d7-1e68b6d4cfa7}
	\tags{definition, integration, 1.3.6}
	\begin{field}
		Sei $M \subseteq \mathbb{R}^n$ eine $k$-dim. UM. Sei $M\rightarrow \mathbb{R}$ eine Funktion.
		Was hei"st es, dass \textit{$f$ "uber $M$ integrierbar} ist?
	\end{field}
\begin{field}
	Es liege einer der folgenden F"alle vor:
	\begin{enumerate}
		\item $\exists$ eine globale Parametrisierung $(\Phi,T)$.
		\item $\Phi : T \rightarrow M$ sei eine lokale Parameterdarstellung, $f$ habe einen kompakten Tr"ager $\text{supp}f \subseteq \Phi\qty[T]$ (d.h. $f \circ \Phi$ hat kompakten Tr"ager in $T$).
	\end{enumerate}
	Dann hei"st \textit{$f$ "uber $M$ integrierbar}, falls	$f\qty(\Phi(t))\sqrt{g(t)}$ "uber $T$ integrierbar ist. In diesem Fall setzt man
	\begin{equation*}
		\int_M f(x)\dif S(x) \coloneqq \int_T f(\Phi(t))\cdot \sqrt{g(t)}\dif t.
	\end{equation*}
\end{field}
	\xfield{Sei $M \subseteq \mathbb{R}^n$ eine $k$-dim. UM. Wie definiert man den \textit{$k$-dim. Inhalt von $M$} ($\abs{M}$)?}
	\begin{field}
		Sei $(\Phi,T)$ eine globale Parametrisierung von $M$. Dann ist 
		\begin{equation*}
			\abs{M} \coloneqq \int_T\sqrt{g(t)} \dif t .
		\end{equation*}
	\end{field}
\end{note}

\begin{note}
	\xplain{cb0f1769-f562-4b8c-85d5-0619ccef3a5e}
	\tags{satz, integration, 1.3.7}
	\begin{field}
		Wie lautet der Satz "uber die Unabhängigkeit der Integration von der Parameterdarstellung?  
	\end{field}
	\begin{field}
		Sei $M \subseteq \mathbb{R}^n$ eine $k$-dim. UM des $\mathbb{R}^n$, seien $(\Phi_1, T_1)$ und $(\Phi_2, T_2)$ 
		lokale Parameterdarstellungen mit $V=\Phi_1[T_2] = \Phi_2[T_2]$, und habe $f: M \rightarrow \mathbb{R}$ kompakten Tr"ager mit $\text{supp}f \subseteq V$.
		Dann gilt
		\begin{equation*}
		\int_{T_1}f\qty(\Phi_1(t)) \cdot \sqrt{g^{(1)}(t)\dif t}=
		\int_{T_2}f\qty(\Phi_2(s)) \cdot \sqrt{g^{(2)}(s)\dif s}
		\end{equation*} 
		Dabei ist $g^{(i)}$ die Determinante des zu $\Phi_i$ geh"orenden Tensors.
	\end{field}
\end{note}

\begin{note}
	\xplain{0691abde-07a7-4d93-acc6-3c22092196b7}
	\tags{definition, satz , 1.3.8, 1.3.9}
	\begin{field}
		Sei $K\subseteq \mathbb{R}^n$ kompakt, $U_1,\dots U_k \subseteq \mathbb{R}^n$ offen mit $K \subseteq \bigcup^k_{j=1} U_j$. Dann $\exists$ die der $U_1 \dots U_k$ untergeordnete Zerlegung der Eins auf $K$. 
		Wie ist sie definiert?
	\end{field}
	\begin{field}
		Die Zerlegung besteht aus Funktionen $\phi_1 \dots \phi_k \in C^\infty_c(\mathbb{R}^n)$ mit folgenden Eigenschaften:
		\begin{enumerate}
			\item $\forall_j \in \qty{1,\dots,k}: \text{supp}\phi_j \subseteq U_j, 0 \leq \phi \leq 1$
			\item $\forall_x \in K: \sum_{j=1}^k \phi_j(x) = 1$
		\end{enumerate}
	\end{field}
	\begin{field}
		Sei $K$ eine Menge, wann ex. eine Partition der Eins auf $K$?
	\end{field}
	\begin{field}
		Wenn $K \subseteq \mathbb{R}^n$ und kompakt und $\exists $ offene Mengen $U_1, \dots U_k \in \mathbb{R}^n$ mit $K \subseteq \bigcup^k_{j=1} U_j$. 
	\end{field}
\end{note}

\begin{note}
	\xplain{df4c7e7c-51ba-4722-81b8-95fbc5596d7b}
	\tags{definition, 1.3.10}
	\begin{field}
		Wie definiert man einen Integral einer  Funktion $f$ "uber eine $k$-dim. UM $M$ des $\mathbb{R}^n$, wenn es keine globale Parametrisierung gibt? Was wird vorausgesetzt?
	\end{field}
		\begin{field}
			Es seien gegeben:
			\begin{itemize}
				\item $f \in C_c^\infty (M)$ $\Rightarrow \exists$
				lokale Parametrierungen $\Phi_j : \mathbb{R}^k \supseteq U_j \rightarrow V_j \subseteq M (j \in \qty{1, \dots, m})$ mit $\text{supp}f\subseteq \bigcap^m_{j=1}V_j$ . 
				
				\item offene Mengen $W_j\in \mathbb{R}^n$ mit $V_j = M \cap W_j$.
				
				\item Eine der "Uberdeckung $W_1 \dots$ zugeordnete Zerlegung der Eins auf $\text{supp}f$:
				$\phi_1, \dots \phi_m$.
			\end{itemize}
		Dann setzt man:
		\begin{equation*}
		\int_M f(x) \dif S(x) \coloneqq \sum_{j=1}^m \int_{V_j} (\phi_j f)(x) \dif S(x).
		\end{equation*}
		\end{field}
\end{note}

\begin{note}
	\xplain{b63f7742-3cb5-48c1-a1e6-ad3c22476935}
	\tags{definition, orientierung, 1.4}
	\begin{field}
		Seien $V$ ein $k$-dim. Vektorraum, $B_1= \qty(v_1, \dots, v_k)$, $B_2 = \qty(w_1, \dots ,w_k)$ zwei Basen von $V$. $B_1$ und $B_2$ hei"sen \textit{gleichorientiert}, falls... 
	\end{field}
	\begin{field}
		\begin{equation*}
		\text{det}A > 0
		\end{equation*}
		wobei $A = (a_{ij})$ "uber
		\begin{equation*}
			\forall_i \in \qty{1,\dots'k} : w_i = \sum_{j=1}^k a_{ij} v_j
		\end{equation*}
		definiert ist.
	\end{field}

	\begin{field}
		Ist $F: \mathbb{R}^n \rightarrow \mathbb{R}^n$ \textit{orientierungserhaltend} und bzgl. der kanonischen Basis durch die Matrix $C$ dargestellt, dann gilt:
	\end{field}
	\xfield{$\text{det}C > 0 $}
	
	\begin{field}
		Wor"uber wird die Orientierung einer UM definiert?
	\end{field}
	\xfield{Tangentialr"aume}
	
	\begin{field}
		Sei $M$ eine $k$-dim. UM des $\mathbb{R}^n$.
		Eine Basis in $T_a(M)$ hei"st positiv orientiert , wenn... 
	\end{field}
	\begin{field}
		sie das Bild einer positiv orientierten Basis in $\mathbb{R}^k$ unter $\Phi ' (c)$ ist. Wobei
		\begin{equation*}
			\Phi : \mathbb{R}^k \supseteq T \rightarrow M \subseteq \mathbb{R}^n
		\end{equation*}
		eine Parametrisierung und $\Phi(c) = a$.
		
		
		Die Basis $\qty(\diffp[]{\Phi}{t_1}(c), \dots)$ ist positiv orientiert. 
	\end{field}
\end{note}

\begin{note}
	\xplain{d67b7f3a-8506-468a-874e-c3612155dbec}
	\tags{definition, orientierung, 1.4.1}
	\begin{field}
		$M$ hei"st \textit{orientierbar}, wenn es...
	\end{field}
	\begin{field}
		ein System $O$ von Karten $(h,W)$ gibt mit:
		\begin{enumerate}	
			\item $\bigcup_{W\in O} W = M$,
			\item $(W_1,W_2 \in O \wedge W_1\cap W_2 \neq \emptyset) \Rightarrow \forall_{a \in W_1 \cap W_2}$
			liefern $(h_1, W_1)$ und $(h_2, W_2)$ die gleiche Orientierung von $T_a(M)$.
	\end{enumerate}	
		
		
	Man sagt auch: F"ur $M$ gibt es eine lokal vertr"agliche Menge von Orientierungen der Tangentialr"aume.  	
	\end{field}
\end{note}

\begin{note}
	\xplain{cb63b56e-44d2-468d-ab19-3bf04cb89092}
	\tags{definition, orientierung, 1.4.2}
	\begin{field}
		Zwei Karten $(h_1, W_1)$, $(h_2, W_2)$ hei"sen gleichorientiert (bzw. der zugeh"orige Kartenwechsel orientierungserhaltend),
		wenn... 
	\end{field}
	\begin{field}
	 	$\text{det}(h_2\circ h_1^{-1})>0$.
	\end{field}
\end{note}

\begin{note}
	\xplain{de2f80aa-14cf-4e46-9bc7-f98c32a2416c}
	\tags{lemma, orientierung, 1.4.3}
		\xfield{Eine $k$-dim. UM $M$ des $\mathbb{R}^n$} ist genau dann orientierbar, wenn...
		\xfield{es einen Atlas aus gleichorientierten Karten gibt.}
\end{note}

\begin{note}
	\xplain{dd73e6ed-29a1-46ba-b883-dfdfa431bcd5}
	\tags{lemma, orientierung, 1.4.4}
	\begin{field}
		Sei $M$ eine $(n-1)$-dim. UM des $\mathbb{R}^n$. Dann ex. ein eindeutige Beziehung zwischen den Orientierungen von $M$ und den stetigen Einheitsnormaleinvektorfeldern auf $M$.
		
		Erkläre den Beweis zu dem Lemma (Beweisidee).
	\end{field}
	\begin{field}
		Sei	$M$ orientierbar, $a\in M$, es g"abe eine  lokale Parametrisierung $(\Phi, T)$ mit $T(c) =a$. $\dim (T_a)= n-1$. 
		
		Also es gibt zwei auf $T_a$ orthogonale Einheitsvektoren (in 1D, unterschied in der Orientierung der Vektoren).
		
		W"ahle $n(a)$ so, dass $(n(a), \Phi'(c)e^{(1)}, \dots, \Phi'(c)e^{(n-1)})$ positiv orientiert in $\mathbb{R}^n$ ist.
		
		Tue das f"ur jeden Punkt im $M$ $\rightarrow$ gesuchtes Vektorfeld.
	\end{field}
\end{note}

\begin{note}
	\xplain{cd74f947-24f9-4fac-a82f-7ffbcd11b50b}
	\tags(definition, rand, 1.5)
	\xfield{Wie sind $\mathbb{R}^k_- \qq{und} \partial\mathbb{R}^k_-.$ definiert? }
	\begin{field}
		\begin{itemize}
			\item $\mathbb{R}^k_- = \qty{(t1_, \dots, t_k)\tran \in \mathbb{R}^k; t_1 \leq 0}$
			\item $\partial\mathbb{R}^k_- = \qty{(t1_, \dots, t_k)\tran \in \mathbb{R}^k; t_1 = 0}$
		\end{itemize}
	\end{field}
\end{note}
\begin{note}
	\xplain{91f8a5f0-f29c-4a29-ad15-46ac7e450eea}
	\tags{definition, rand, 1.5.1}
	\begin{field}
		$M \subseteq \mathbb{R}^n$ ist eine $K$-dim. UM mit Rand, wenn ... 
	\end{field}
	\begin{field}
		es um jeden Punkt $p \in M$ eine lokale Parameterdarstellung $(\Phi, T)$ mit $p \in \Phi[T]$ und $T$ offen im $\mathbb{R}^k_-$
	\end{field}
	\begin{field}
		$M \subseteq \mathbb{R}^n$ ist eine $K$-dim. UM mit Rand. $(\Phi, T)$ eine lokale Parametrisierung. Der Punkt $p \in M$ hei"st Randpunkt von $M$, wenn...
	\end{field}
	\begin{field}
		$p = \Phi(t)$ mit $t \in T\cap \partial\mathbb{R}^k_-$. 
	\end{field}
	\begin{field}
		Eine Parameterdarstellung $(\Phi, T)$ hei"st \textit{randadaptiert} falls...
	\end{field}
	\begin{field}
		\begin{equation*}
			T\cap \partial\mathbb{R}^k_- \neq \emptyset
		\end{equation*}
	\end{field}
\end{note}
\begin{note}
	\xplain{ff11fe09-aac0-456a-b8bf-f652b973a4d7}
	\tags{satz, definition, rand, 1.5.2}
	
	\begin{field}
		Sei $M \subseteq \Rn$ eine $k$-dim. UM mit Rand. Dann gilt: (zwei Aussagen)
	\end{field}
	\begin{field}
		\begin{enumerate}
			\item $\partial M$ ist eine k-1 dimensionale UM ohne Rand.
			\item $M$ orientierbar $\Rightarrow$ $\partial M$ orientierbar.
		\end{enumerate}
	\end{field}
	\begin{field}
		Wann ist $\partial M$ positiv orientiert?
	\end{field}
	\begin{field}
		Seien $p\in \partial M $ und $(\Phi,T)$ randadaptiert. Eine Basis ($B$) von $T_p(\partial M)$ sei genau dann positiv orientiert, wenn die Basis (v|B) in $T_p(M)$ es ist. Wobei
		\begin{equation*}
			v \coloneqq \Phi'(t)e^{(1)} \in T_p(M)
		\end{equation*}
	\end{field}
\end{note}
\begin{note}
	\xplain{5f486117-567d-446d-9867-72a8ebba6ac0}
	\tags{definition, gauss, 1.5.3}
	\begin{field}
		Wie definiert man einen regul"aren und einen singul"aren Randpunkt? Wie ist der Normaleneinheitsvektor definiert?
	\end{field}
	\begin{field}
		Sei $G$ ein Gebiet so, dass $B \bar{G}$ kompakt ist.Ein Punkt $a \in \partial B$ hei"st regul"arer Randpunkt von $B$, wenn
		es eine offene Umgebung $U$ um $a$ gibt und $g: U \rightarrow \mathbb{R}$ stetig mit:
		\begin{enumerate}
			\item $B \cap U = \qty{x \in U; \ g(x) \leq 0}$,
			\item $ \forall_x \in U : \text{grad} \ g(x) \neq 0$.
		\end{enumerate}
	Menge aller regul"aren Randpunkte in $\partial B$ wird mit $\partial_r B$ bezeichnet.
	 $(a \in \partial B \wedge a\notin \partial B_r) $, dann hei"st $a$ singul"arer Randpunkt.
	  Analog ist $\partial_s B = \partial B \setminus \partial_r B$.
	 \begin{equation*}
	 	n(a) \coloneqq \frac{\text{grad}u g(a)}{\norm{\text{grad}g(a)}}
	 \end{equation*}
	 ist der ("au"sere) Normaleineinheitsvektor an $\partial B$ in $a$.
	\end{field}
	\begin{field}
		Ein Teilmenge $G$ hei"st Gebiet, falls...
	\end{field}
	\begin{field}
		es offen, nichtleer und zusammenh"angend ist.
	\end{field}
	\begin{field}
		Eine Teilraum ist zusammenh"angend, falls...
	\end{field}
	\begin{field}
		 es nicht als Vereinigung zweier nichtleerer getrennter Mengen geschrieben werden kann. 
		 
		 Es gibt viele "aquivalente Definitionen. 
	\end{field}
	\begin{field}
		Sei $B$ kompakt. Sei $\partial_s B = \emptyset$ dann hei"st $B$...
	\end{field}
	\xfield{Kompaktum mit glatten Rand.}
	\begin{field}
		Was ist ein Kompaktum mit glatten Rand?
	\end{field}
	\begin{field}
		Es ist eine kompakte Menge  ($M$) mit $\partial_s M = \emptyset$. 
	\end{field}
\end{note}

%Satz von Gauss
\begin{note}
	\xplain{50609bfb-331c-4a5c-a4a7-47e4b3d882a1}
	\tags{satz, gauss, stokes,  1.5.4, 1.5.5}
	\xfield{Wie lautet der Satz von Gau"s?}
	\begin{field}
		Seien $B\subseteq \Rn$ ein Kompaktum mit glatten Tand, $n : \partial B \rightarrow \Rn$ das "au"sere Einheitsnormalenfeld,
		$F:B \rightarrow \Rn$. Dann gilt:
		\begin{equation*}
			\int_B \text{div}F(x)\dif x = \int_{\partial B} \left<F(x), n(x)\right>\dif S
		\end{equation*}
	\end{field}
	\xfield{Wie lautet der klassische Satz von Stokes?}
	\begin{field}
		Sei $M\subseteq \mathbb{R}^3$ eine kompakte 2-dimensionale UM mit Rand $\partial M$. $M$ sei durch ein Einheitsnormalenvektorfeld   $n: M \rightarrow \mathbb{R}^3$ orientiert. $\partial M$ habe die von $M$ induzierte Orientierung. 
		$t: \partial M \rightarrow \mathbb{R}^3$ bezeichne das Tangenteneinheitsfeld an die Kurve $\partial M$. Sei $F: M \rightarrow \mathbb{R}^3$ ein stetig diff'bares Vektorfeld. Dann gilt:
		\begin{equation*}
			\int_M \left<\text{rot} F, n\right>\dif S = \int_{\partial M} \left<F, t\right> \dif s
		\end{equation*}
	\end{field}
\end{note}
\end{document}
